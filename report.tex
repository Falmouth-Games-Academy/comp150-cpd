% Please do not change the document class
\documentclass{scrartcl}

% Please do not change these packages
\usepackage[hidelinks]{hyperref}
\usepackage[none]{hyphenat}
\usepackage{setspace}
\doublespace

% You may add additional packages here
\usepackage{amsmath}

% Please include a clear, concise, and descriptive title
\title{Improving My Software Development Skills}

% Please do not change the subtitle
\subtitle{COMP150 - CPD Report}

% Please put your student number in the author field
\author{1806868}

\begin{document}

\maketitle

\section{Introduction}

I am looking to become a computing profession and game developer, whit a focus on audio programming and AI/Machine learning I also have an interest in open source software and one day would like to create an open source game project. To achieve this there is several skills that I need to continuously improve on. I would like to be able to do more in-depth code reviews to better support my team, dive deeper in to math by induction so I can fully understand it and its benefits. I also need to improve my research organization skills, so I can write in more depth about the subject, I need to gain more confidence while talking during presentations to prevent me from panicking and forgetting my lines when it’s my time to talk. I feel that I would also benefit from learning how to deal with conflict with in a team as currently I prefer to take a step back which does not help to resolve the issue.

\section{Code Review}

Throughout the group development project, I found that it was hard to do an in-depth code review and often found that I missed stuff that I notices later. To achieve becoming a successful software developer, I need to be able to give clear and precise recommendation to aid my colleges. This has resulted in me missing obvious bugs that have been difficult to find further down the line. With a bit practice, learning some new techniques I can overcome this to better guide my team in the future. 

In-order to improve this I will read “Why Code reviews matter” \cite{codeReviewsMatter}, “10 ways to improve your code reviews” \cite{waysToImprove} and read others code reviews on StackExchange.com \cite{stackexchange}. After I have reviewed my team mates code, I will ask them for feedback to see if it helped them in any way. I am already able to give basic feedback, but I want to be able to give richer and more precise feedback to improve the quality of our code and better guide my team in the future. I am aiming to gain these skills by the end of our next game project.


\section{Math By Induction}

In comp-110 I found that math by induction was extremely challenging. As an aspiring computing professional, I would benefit from mastering induction to aid me in proving that my algorithms are correct. However, I seem to get lost half way through the equations which results in me getting stressed because I don’t fully understand how it works. I know that it can be achieved in several simple steps, but I am having difficulty recalling the order of them. I know that I can overcome this challenge if I put more focus into it and go at a slower pace to make sure that I fully understand procedures step by step.

To correct my understanding, I will complete a short math course on Udemy \cite{Udemy} which covers math induction. Once I have completed the induction lessons, I can book a meeting with a tutor to check that I can implement the skills correctly. I understand why induction is useful to prove that a math equation is correct. This will allow me to check that the math behind my algorithm is correct before I begin programming it. I intend to have achieved this by the end of the second semester. 

\section{Third Key Skill}

Write about 200 words. As above.

\section{Fourth Key Skill}

Write about 200 words. As above.

\section{Fifth Key Skill}

Write about 200 words. As above.

\section{Conclusion}

Write your conclusion here. Though the conclusion should be brief, no more than 100 words, it should do more than merely summarise the report. Focus on the five SMART actions that you intend to take in order to overcome any challenges and/or obstacles. Contextualise how this will help you towards your intended career goal and how this may improve your project for the next semester.

\bibliographystyle{IEEEtran}
\bibliography{references}

\end{document}