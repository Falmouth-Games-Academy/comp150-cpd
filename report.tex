% Please do not change the document class
\documentclass{scrartcl}

% Please do not change these packages
\usepackage[hidelinks]{hyperref}
\usepackage[none]{hyphenat}
\usepackage{setspace}
\doublespace

% You may add additional packages here
\usepackage{amsmath}

% Please include a clear, concise, and descriptive title
\title{Adventures in Personal Development}

% Please do not change the subtitle
\subtitle{COMP150 - CPD Report}

% Please put your student number in the author field
\author{1604281}

\begin{document}

\maketitle

\section{Introduction}

Although I still do not have a single, solid career goal, I aim to work in the computing industry. To this end, I am currently working to equip myself with relevant computing skills, particularly practical programming experience. During my time at university so far, I have identified five key skills that I could be improved to benefit my programming, especially within a group development environment.

These skills are: Dealing with Frustration, Team Communication and Direction, Unbiased Time Management, Game-Specific Programming Knowledge and Problem Identification.

\section{Dealing with Frustration}

Dealing with frustration when code doesn't work and turning that frustration into productive problem solving. Issues often arise during any form of programming and Game Development is no exception. Spending valuable time getting worked up over a problem can be a significant detriment to the development process.

\paragraph{Smart Action} When encountering a frustrating problem, I will spend no more than half an hour on this problem before taking a break and working on something else. When I return to the problem, I may notice something that I did not before. In addition, I will make better use of the resources available to me - the library, lectures and peers - to help me resolve programming problems, rather than always attempting to fix the issue myself.


\section{Team Communication and Direction}

During a group project, keeping the whole team in contact and participating in development is vital. Some team members can become somewhat estranged from the development process if left alone and good management and communication is required to ensure everyone is contributing and finding something they are able to participate in. In a game development environment, it is crucial that intra-team communication is clear and constant as there are often multiple teams working on different aspects of the game. 

During my last collaborative game design project some team members did not feel confident engaging with the project and therefore drifted away. Although there was no designated project leader, as one of the people who recognised this issue, I could have done more to keep the team a cohesive unit. Assigning specific tasks for team members who felt directionless for example. Our team communication also lacked in some regards as meetings were occasionally poorly planned and people did not always receive the meeting times when they should have. This impacted our work as often people did not always have a clear idea of what we were working on next or how they should proceed.

\paragraph{Smart Action} I will ensure I examine the Github commit log once a week to ensure everyone has participated fully. I will also make sure to contact every member of the team once per week to ensure they are happy with their current tasks and feel that they are making meaningful contributions. Ensuring that the team knows what they should be doing and have the means to do so will help the project progress more smoothly.


\section{Unbiased Time Management}

Lack of time management for essays compared to programming. As I find programming more interesting I tend to spend more time on programming tasks compared to essay tasks. This is obviously detrimental to university work, however, in a game development context, written reports and documents can be important aspects of the development process.

My poor time management was particularly noticeable when I had two essays to write while also working on a collaborative game development project. I spent a lot of time working on the game early on - rather than working on the essay drafts - which meant that I later had to forgo working on the game to catch up on essay writing. Had I handled my time better I could probably have achieved better quality essays as well as potentially polishing some game features better.

\paragraph{Smart Action} When I have essay tasks, I will ensure that I dedicate at least one day per week to focusing solely on the essay task. I will also ensure that I begin this as soon as the essay is set so that I can be slowly but constantly working on it. This will allow me the time to focus on game development while not neglecting my essay writing and rushing it later.


\section{Game-Specific Programming Knowledge}

Although I have a modicum of programming skill, I am fairly new to the application of programming in a game development environment. Because of this, I have a lack of knowledge concerning common game-related features such as the use of vectors, delta-time and sprite transformations. A better understanding of these concepts, as well as other game-specific skills, especially the use of the Unreal Engine, would greatly improve my ability to contribute to game development projects and allow me to progress faster, without having to constantly research new concepts.

\paragraph{Smart Action} I will spend fifteen minutes a day researching game-related programming concepts, either by reading documentation and tutorials related to game design or by examining source code from example game projects. This will help further my knowledge of core game development features.


\section{Problem Identification}

The ability to more quickly identify the origin of programming problems would allow me to spend less time testing possible issues and more time fixing the actual problem. In a development environment - where there are deadlines to be met - being able to fix problems quickly would greatly increase my efficiency. A large proportion of the time in any development project is spent dealing with bugs and errors, and this is particularly true of game development, where a single feature may be play-tested multiple times and tweaked constantly. Being able to fix small but critical issues fast would therefore be a useful skill. 

Currently I feel that I often take too long to realise where a problem originates. This is likely due in part to my lack of knowledge in various areas but more specifically, my rudimentary use of debugging tools. 

\paragraph{Smart Action} I will spend half an hour a week familiarising myself with the documentation of the Unreal debug suite to be more confident using the debugger to identify problems. I will then make sure to utilise the debug tools during any project to allow me to better identify issues.


\section{Conclusion}

The key skills highlighted above are just five of the areas that I feel I could improve in. However, they are in my opinion, five of the most important things to improve with regards to helping me become a better programmer. The simplest SMART actions are those for Problem Identification and Game-Specific Programming Knowledge, as they mainly consist of research that can be done in my own time, while Dealing with Frustration is largely a mental exercise. 

Unbiased Time Management is one of the most important, as it can influence the quality of all my work. However it is also a fairly straight-forward action to work on, provided I can set and adhere to a schedule. Team Communication and Direction is a far more complex action at its heart, as it revolves around maintaining a healthy group environment and interacting closely with team members to ensure they are comfortable participating. Despite this, I have set a short, concise SMART action which will hopefully help to make this task easier.


\bibliographystyle{ieeetran}
\bibliography{references}

\end{document}
