% Please do not change the document class
\documentclass{scrartcl}

% Please do not change these packages
\usepackage[hidelinks]{hyperref}
\usepackage[none]{hyphenat}
\usepackage{setspace}
\doublespace

% You may add additional packages here
\usepackage{amsmath}

% Please include a clear, concise, and descriptive title
\title{Continuing Professional Development}

% Please do not change the subtitle
\subtitle{COMP150 - CPD Report}

% Please put your student number in the author field
\author{1806384}

\begin{document}

\maketitle

\section{5 Key Skills That I Consider Obstacles}
The first key skill that I considered to be an obstacle was my Python programming ability when I started this project.  The second key skill that I considered to  be  an  obstacle.   The  second  key  skill  that  I  considered  to  be  an  obstacle was working in teams on a project for the first time.  The third key skill that I considered to be an obstacle was my knowledge of Agile Game Development with Scrum.  The fourth key skill that I considered to be an obstacle was my knowledge of GitHub and ability to use GitHub efficiently.  The fifth key skill that I considered to be an obstacle was that I wasn't committing often enough which made my commits bigger than they needed to be.

\section{The  relevance  and  importance  of  these  five key skills}
The importance of the first key skill that I mentioned; my python programming ability is that the game itself was programmed in Python therefore I needed to learn Python quickly so I could be able to progress on the game and code the features my team needed me to. Python is relevant to the game because the game was programmed in Pygame which using Python. The importance of the second key skill that was mentioned is that I had to learn to work in teams as we were thrown into our team relatively quickly, so we had to adjust and assign roles to ourselves that we thought we could manage. I think I adjusted to working in teams quickly but at first it was challenging. The relevance of working as a team is that we needed to work as one to be able to complete our game which we managed to do so it turned out very well. The importance of the third key skill that I mentioned was that we were using Agile Game Development with Scrum to guide us on our game development, but we had never used this before so it was new to all of us, we did make some mistakes along the way as my knowledge of it was severely lacking but along the way I got used to it and increased my knowledge on this development style, this was relevant to the project as it was a guide to how we should be developing our game. The importance of the fourth key skill I mentioned is that version control is a major asset when developing games in teams and the fact I hadn’t used version control before made it hard to get used to committing frequently and reviewing my teammates work, but I definitely improved my knowledge of version control whilst working on this project I think in the end our commits and reviewing of each other's work was a lot better than what it was when we first started the project, this key skill is relevant to working on games as version control is a must in game development as it makes implementing your code into the project a lot smoother than it would be without version control. The importance of the fifth key skill I mentioned is that committing regularly makes it so you can easily go back a version if you encounter a bug with your code, if you don’t commit regularly it’ll be a lot harder to find this bug as you’ll be committing a must larger portion of code than if you were committing every small feature you code. This is a good habit to get into because there are no drawbacks to committing regularly and it’s a lot easier for your teammates to see what you’ve added to the game if there’s lots of small commits with the features instead of one large commit, this is relative to game development because committing regularly is imported when working with other people because it’ll make their life easier if they can see what you added, and it’ll make your life easier when you encounter a bug. 

\section{Assessing my application of each of these skills}
At first I struggled with the application of the Python programming but I learned Python as I went along which helped me to advance with the set tasks my team had assigned me so I definitely made progress with this key skill as I was able to complete the tasks needed to get the game to where it is today. I had to work in teams which I hadn’t done before in this style of game development which was tricky at first but once our team sat down with each other we began to work out a game plan which helped us get started on our game, fortunately none of us clashed so we could get straight on with the game but because it was my first time working in a team I was reluctant to ask for help which wasn’t good, I should’ve asked for help when I needed it. I had never done Agile Game Development before so this was a new experience to me, I didn’t know every aspect of this game development therefore I struggled to comprehend every factor of Agile but as the project went along I acquired the information I didn’t have before making me comprehend it completely so I’ll be able to use it efficiently in the next projects I work on. I had to use GitHub for this project which I had never used before, starting to use this when I had never before was tricky as I had to get used to using it so I could commit regularly this was an issue because I had so many other things to focus on like learning Python more and programming the game as well as trying to learn GitHub it made time management difficult. I had to force myself to try and commit regularly but I always seemed to commit large portions of code instead of small commits in the final week of this project it finally stuck in my head to commit every little thing I did but if I got it in my head sooner the project would have flowed a lot smoother but at least I now know that committing regularly is really important and I will be doing that for future projects.

\section{How to overcome these obstacles using SMART actions}
The first obstacle I encountered I overcome quickly but if I had to do it all over again, I would have pre planned learning Python efficiently so I could just straight into the project. The second obstacle is very different to the first obstacle you can’t really plan to be better working in teams but how to overcome this obstacle would be spending time getting to know your team and making sure there’s nothing that you clash on. I could have read up on Agile Game Development with Scrum before I started this project which would have been good use with my time which I wish I had done. I could have overcome four of my obstacles my setting a time table to learn about each one of them, one day a week for Python, another day for Agile, another day for GitHub and the other day for regular commits this would have helped me overcome these obstacles drastically and I would have been able to make a huge difference on this project.


\bibliographystyle{ieeetran}
\bibliography{references}

\end{document}
