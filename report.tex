% Please do not change the document class
\documentclass{scrartcl}

% Please do not change these packages
\usepackage[hidelinks]{hyperref}
\usepackage[none]{hyphenat}
\usepackage{setspace}
%\doublespace so I can read it

% You may add additional packages here
\usepackage{amsmath}

% Please include a clear, concise, and descriptive title
\title{A Summary of Personal Experiences in Falmouth Adversity}

% Please do not change the subtitle
\subtitle{COMP150 - CPD Report}

% Please put your student number in the author field
\author{1707981}

\begin{document}

\maketitle

\section{Introduction}
Oops I meant University.

When I arrived here I was expecting something entirely new and scary. What I got was much more, and it pushed many of my flaws to the surface. \textbf{Managing anxiety (and not spreading it!)}, \textbf{managing friendships and acquaintanceships}, \textbf{assertion and conflict}, \textbf{reducing focus} and some other noticeably autistic flaws. Most of these are weaknesses I knew I had. However, in my inherent optimism I have been working hard to spot them and wipe them out, to my strengths of \textbf{overdedication} and \textbf{inappropriate creativity}. I've seen better strengths.

Here lies the summary of a socially oblivious autistic adult's foray into the social world of drinking, programming and writing essays alternately.

Write your introduction here. A brief introduction of about 100 words is recommended, which should state your career goal and the five key skills that you wish to highlight from your weekly reports. When choosing which skills to focus on for this report, be specific. Avoid choosing broad skills that are clearly important for any student, such as \textit{time management} or \textit{communication}. Instead, make it more granular. Consider which specific aspects of these broad areas are a priority for you, personally, and what may have caused or exacerbated the challenge. Tutors are not assessing your knowledge of general study skills. Rather, they are assessing your ability to analyse and reflect on your own learning and personal development as an individual and towards becoming a computing professional.

\section{First and Foremost - Anxiety}
The professional world is a social place. It is work-ridden as well. Social anxiety is a massive pain, and it leaks into other areas like time management and creativity, for reasons of 1) difficulty relaxing and socialing between work and 2) fear of judgement.

Write about 200 words about. Remember, this is should be reflective and personal to you. Justify the relevance and importance of each of these skills with insight into your personal goals and personal circumstances. Assess your application of the skill throughout the semester and critically reflect on upon their impact it has had on your work and the challenges/obstacles. Acknowledge difficulties. Then, suggest how to overcome the challenge/obstacle in relation to a SMART action. When planning such actions, do not be too general. Consider SMART actions:
specific; measurable; achievable; relevant; and time-bound. Ensure that your proposed action for future development meets all five of these criteria.

\section{Second Key Skill}

Write about 200 words. As above.

\section{Third Key Skill}

Write about 200 words. As above.

\section{Fourth Key Skill}

Write about 200 words. As above.

\section{Fifth Key Skill}

Write about 200 words. As above.

\section{Conclusion}

Write your conclusion here. Though the conclusion should be brief, no more than 100 words, it should do more than merely summarise the report. Focus on the five SMART actions that you intend to take in order to overcome any challenges and/or obstacles. Contextualise how this will help you towards your intended career goal and how this may improve your project for the next semester.

\bibliographystyle{ieeetran}
\bibliography{references}

\end{document}
