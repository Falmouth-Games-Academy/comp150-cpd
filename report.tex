% Please do not change the document class
\documentclass{scrartcl}

% Please do not change these packages
\usepackage[hidelinks]{hyperref}
\usepackage[none]{hyphenat}
\usepackage{setspace}
\doublespace

% You may add additional packages here
\usepackage{amsmath}

% Please include a clear, concise, and descriptive title
\title{Personal Development Report}

% Please do not change the subtitle
\subtitle{COMP150 - CPD Report}

% Please put your student number in the author field
\author{1604281}

\begin{document}

\maketitle

\section{Introduction}

Write your introduction here. A brief introduction of about 100 words is recommended, which should state your career  goal and the five key skills that you wish to highlight from your weekly reports. When choosing which skills to focus on for this report, be specific. Avoid choosing broad skills that are clearly important for any student, such as \textit{time management} or \textit{communication}. Instead, make it more granular. Consider which specific aspects of these broad areas are a priority for you, personally, and what may have caused or exacerbated the challenge. Tutors are not assessing your knowledge of general study skills. Rather, they are assessing your ability to analyse and reflect on your own learning and personal development as an individual and towards becoming a computing professional.

\section{Dealing with Frustration}

Dealing with frustration when code doesn't work and using turning that frustration into productive problem solving.

\paragraph{Smart Action} When encountering a frustrating problem, I will spend no more than half an hour on this problem before taking a break and working on something else. When I return to the problem, I may notice something that I did not before.


\section{Team Communication and Direction}

Keeping whole team in contact and participating in develpoment. Some team members became somewhat estranged from the develpment process and better management and communication on my part could help ensure everyone is contributing and finding something they are able to participate in.

\paragraph{Smart Action} I will ensure I exaiming the github commit log once a week to ensure everyone has participated fully. I will also make sure to contact every member of the team once per week to ensure they are happy with their current tasks and feel that they are making meaningful contributions.


\section{Unbiased Time Management}

Lack of time managment for essays compared to programming. More interested in programming so spend more time on progrmaming tasks than essay tasks.

\paragraph{Smart Action} When I have essay tasks, I will ensure that I dedicate at least one day per week to focusing solely on the essay task.


\section{Game-Specific Programming Features}

Better knowledge of game-specific or game-relevant programming aspects such as vectors, deltatime, etc

\paragraph{Smart Action} I will spend 15 minutes a day researching game-related programming concepts.


\section{Problem Identification}

The ability to more quickly identify the origin of programming problems. 

\paragraph{Smart Action} I will spend half an hour a week familiarising myself with the documentation of the pycharm debug suite to be more confident using the debugger to identify problems.


\section{Conclusion}

Write your conclusion here. Though the conclusion should be brief, no more than 100 words, it should do more than merely summarise the report. Focus on the five SMART actions that you intend to take in order to overcome any challenges and/or obstacles. Contextualise how this will help you towards your intended career goal and how this may improve your project for the next semester.

\bibliographystyle{ieeetran}
\bibliography{references}

\end{document}
