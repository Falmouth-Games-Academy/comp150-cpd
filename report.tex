% Please do not change the document class
\documentclass{scrartcl}

% Please do not change these packages
\usepackage[hidelinks]{hyperref}
\usepackage[none]{hyphenat}
\usepackage{setspace}
%\doublespace so I can read it

% You may add additional packages here
\usepackage{amsmath}

% Please include a clear, concise, and descriptive title
\title{A Summary of Personal Experiences in Falmouth Adversity}

% Please do not change the subtitle
\subtitle{COMP150 - CPD Report}

% Please put your student number in the author field
\author{1707981}

\begin{document}

\maketitle

\section{Introduction}
Oops I meant University.

When I arrived here I was expecting something entirely new and scary. What I got was much more, and it pushed many of my flaws to the surface. \textbf{Managing anxiety (and not spreading it!)}, \textbf{managing friendships and acquaintanceships}, \textbf{assertion and conflict}, \textbf{reducing focus} and some other noticeably autistic flaws. Most of these are weaknesses I knew I had. However, in my inherent optimism I have been working hard to spot them and wipe them out, to my strengths of \textbf{overdedication} and \textbf{inappropriate creativity}. I've seen better strengths.

Here lies the summary of a socially oblivious autistic adult's foray into the social world of drinking, programming and writing essays alternately.

\textit{Note: During this report I may make outlandish-looking claims about subjective things. This is to keep my word count with words like 'probably', but this is always implicit. These are holes in my skills, so I don't I have the mindset to find the right answers immediately.}

Write your introduction here. A brief introduction of about 100 words is recommended, which should state your career goal and the five key skills that you wish to highlight from your weekly reports. When choosing which skills to focus on for this report, be specific. Avoid choosing broad skills that are clearly important for any student, such as \textit{time management} or \textit{communication}. Instead, make it more granular. Consider which specific aspects of these broad areas are a priority for you, personally, and what may have caused or exacerbated the challenge. Tutors are not assessing your knowledge of general study skills. Rather, they are assessing your ability to analyse and reflect on your own learning and personal development as an individual and towards becoming a computing professional.

\section{My anxiety and its impact on others}
During the LEGO Robot Olympiad, I struggled with one particularly strong personality. The base issue was that he had predetermined which work needed doing and how, without giving us many opportunities to contribute. When the opportunities did come about I felt alienated and useless.

A different, but related issue was in pair programming. My pair programmer and I had a skill gap favouring me, and I sometimes had trouble keeping our spirit and motivation up. It went both ways--he occasionally expressed that he hated not knowing the answers to things, and expected better of himself, rather than disliking my navigating him.

Both were affected by our personalities. The root problems on my side were my assertion and presentation. I think my anxiety is too noticeable, so one person may think I'm clueless, or another might "catch" the same anxiety.

To improve my inspirational effect on others, I'll take structured notes on my phone every Sunday, outlining the most anxious social moment of the week. Categories will be `the conversation', `why was I anxious', `did they catch it', 'how did it compare to last time', 'how I will rewrite the conversation'. In the comparison, I'll compare my new approaches to the previous, to see whether it helped. Meanwhile, the other sections--especially the rewrite--will help me create an approach for next time.

\section{Problem management}

In classic programmer fashion, the team game project gave me many problems that I spent too many hours focused in on. One of them was collision detection, involving the intersection of two rotatable rectangles. The problems I knew I had to solve piled up instantly, beginning from rotating vectors, to detecting line intersection, to converting tile/pixel units, to managing the hilarious(ly painful) floating point conversions, to shunning the sheer inefficiency of it all. My approach was flawed, as I was writing codependent functions that couldn't be used until the rest were available. This made iteration impossible.

This doesn't suit my learning style, which is usually a hands-on process of experimenting and building upon it.

Next time, my approach should be iterative. Produce a small result with a lazy, baseline structure (e.g., don't check boxes for now, just check lines), then implement the advanced components later. 

I resolved this by coming up with a 'None' solution where in 'None' is an acceptable value for these classes. This sucked. But it allowed the programmer not only to forget to instantiate collision, but not to bother instantiating it at all when an object wasn't solid. The rationale was that we couldn't afford a miscommunication in the team should the team choose to make a new object subclass.

SMART plan:
My main challenge was identifying good programming practice. The best resolution is to discuss it with the team. Therefore, I will attempt to initiate a code review meeting this Friday on the 26th November 2017. During this, I will present to my team the challenge, and ask each member how they would deal with it, individually, ensuring the opinion is purely their own. I will note down their suggestion. Then, I will critique their commentary, asking how they would address a) team members forgetting to instantiate the class, b) performance, c) ease of implementation, d) line count, etc. My measurement of success will entirely depend on whether a single solution has been accepted by the entire team--that may not happen, but will still justify the exploration of alternatives, and I will have formed a personal opinion from the various techniques.

\section{Third Key Skill}

Write about 200 words. As above.

\section{Fourth Key Skill}

Write about 200 words. As above.

\section{Fifth Key Skill}

Write about 200 words. As above.

\section{Conclusion}

Write your conclusion here. Though the conclusion should be brief, no more than 100 words, it should do more than merely summarise the report. Focus on the five SMART actions that you intend to take in order to overcome any challenges and/or obstacles. Contextualise how this will help you towards your intended career goal and how this may improve your project for the next semester.

\bibliographystyle{ieeetran}
\bibliography{references}

\end{document}
