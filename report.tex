
% Please do not change the document class
\documentclass{scrartcl}

% Please do not change these packages
\usepackage[hidelinks]{hyperref}
\usepackage[none]{hyphenat}
\usepackage{setspace}
\doublespace

% You may add additional packages here
\usepackage{amsmath}

% Please include a clear, concise, and descriptive title
\title{Your Title Here}

% Please do not change the subtitle
\subtitle{COMP150 - CPD Report}

% Please put your student number in the author field
\author{DO NOT WRITE YOUR NAME\\Put your student number (on your ID card) here}

\begin{document}

\maketitle

\section{Introduction}

Write your introduction here. A brief introduction of about 100 words is recommended, which should state your career goal and the five key skills that you wish to highlight from your weekly reports. When choosing which skills to focus on for this report, be specific. Avoid choosing broad skills that are clearly important for any student, such as \textit{time management} or \textit{communication}. Instead, make it more granular. Consider which specific aspects of these broad areas are a priority for you, personally, and what may have caused or exacerbated the challenge. Tutors are not assessing your knowledge of general study skills. Rather, they are assessing your ability to analyse and reflect on your own learning and personal development as an individual and towards becoming a computing professional.

The overarching goal of my studies in computing for games is currently to become well versed enough with programming to one day be able to either create my own games from scratch on my own or within a small indie group as a developer. I think of myself as a creative person and would like to gain the technical skills and diligence required to become a professional game developer in the future. My five key skills I need to focus on are:
Stress Management With Academic Writing
Communicating Ideas In A Group
Version Control Management
Self Motivated Learning
Logical Problem Solving


\section{Stress Management With Academic Writing}

Write about 200 words about. Remember, this is should be reflective and personal to you. Justify the relevance and importance of each of these skills with insight into your personal goals and personal circumstances. Assess your application of the skill throughout the semester and critically reflect on upon their impact it has had on your work and the challenges/obstacles. Acknowledge difficulties. Then, suggest how to overcome the challenge/obstacle in relation to a SMART action. When planning such actions, do not be too general. Consider SMART actions:
specific; measurable; achievable; relevant; and time-bound. Ensure that your proposed action for future development meets all five of these criteria.

Affective domain - Stress over assignments or presentations: speaking in front of peers.

The majority of the stress I have felt has come from a lack of confidence in my own abilities to do the work well which makes me put it off until I have to do it. Then my motivation to finish something comes mainly from a fear of failure. This is a very negative way to respond in life in general and especially work life. This was most evident in weeks seven and eight in which we had to hand in two academic papers in short succession. Having not done any form of essay writing for at least ten years I felt very apprehensive when faced with this new challenge, so much so that I waited until the weekend before it needed to be handed in to even start them. 
My SMART goal to curtail this stressful environment that I place myself in at time would be this: for my next academic essay or writing, I will begin any research of the topic as soon we are given it. I will spend at least the equivalent of 3 hours a week on the research or essay until I have a draft and will complete it at least 3 days before the deadline.

\section{Communicating Ideas In A Group}

Interpersonal Domain - Week 2 Team project, trouble getting my views across, too shy.

Working in close knit teams has been new and challenging for me since week 0. I feel that my introvert nature is a hindrance to team communication somewhat, this was evidenced throughout our first group project. During the project the team very rarely communicated well as the project went forward and I felt unable and too uncomfortable to speak with team members at times where some constructive feedback may have been useful. Also, when I did make any design changes that another team member didn't like I would back down immediately rather than defend my position on why the change was made or seek out a compromise. This was also an issue in my first pair programming assignment. I felt too anxious and perhaps too indecisive to speak up arrange to meet up and work on things if I didn't think enough work was being done on the task. My SMART goal to combat this would be to make a conscious effort to provide a new idea on how to change or improve something in our game or team every sprint review in my new development team.


\section{Version Control Management}

Dispositional Domain - Leaving essay writing to the last minute, not committing regularly enough to GitHub.

During our project I would sometimes work on our game on my own at home and not knowing GitHub very well I would save different versions of the game to my local computer without committing it. This made it more difficult merging my own changed when other members had already changed something which affected my code. In some ways the lack of contribution from other team members on GitHub made it more simple to use so I'm slightly concerned about how much more difficult it will be when we start working in our new group of ten. I now know how vital using and having a firm grasp of version control is in the games industry, particularly in large teams. When considering a SMART goal for this, I know that committing to a form of version control like GitHub is something I need to be much more comfortable with and as such I intend to make learning it a priority next semester. I intend to go through the further learning exercises on version control before we start our next project as well as get a head start in learning SVN which our group will most likely be using. I will make an SVN account and set up my own project to test out how everything works before the team starts using it.

\section{Self Motivated Learning}

Cognitive Domain - Just learning in lessons, need to practice more outside of lessons, getting a better understanding of general rules in coding.



\section{Fifth Key Skill}

Procedural Domain - Identifying problems to make solving them easier.

\section{Conclusion}

Write your conclusion here. Though the conclusion should be brief, no more than 100 words, it should do more than merely summarise the report. Focus on the five SMART actions that you intend to take in order to overcome any challenges and/or obstacles. Contextualise how this will help you towards your intended career goal and how this may improve your project for the next semester.

\bibliographystyle{ieeetran}
\bibliography{references}

\end{document}
