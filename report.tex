% Please do not change the document class
\documentclass{scrartcl}

% Please do not change these packages
\usepackage[hidelinks]{hyperref}
\usepackage[none]{hyphenat}
\usepackage{setspace}
\doublespace

% You may add additional packages here
\usepackage{amsmath}

% Please include a clear, concise, and descriptive title
\title{My Continuing Professional Development Report}

% Please do not change the subtitle
\subtitle{COMP150 - CPD Report}

% Please put your student number in the author field
\author{1703086}

\begin{document}

\maketitle

\section{Introduction}

My current long-term career goal is to become a lead game developer in a Indie studio in which I have a lot of creative freedom, and some day be able to develop my own personal game ideas into actual games with a team. My current academic development goals are to become overall a better programmer and games developer, to learn as much about game development as possible and to work more effectively with a team. The 5 key skills that I have identified from the first semester and that need further improvement are:
\begin{small}
\begin{itemize}
  \item Managing stress levels
  \item Planning work with others
  \item Managing my own workload
  \item Self-learning
  \item Applying my knowledge to current projects
\end{itemize}
\end{small}
Improving these skills will greatly help me in becoming a better games developer and team player.

\section{First Key Skill: Managing stress levels}

In weeks 7 and 8 of the semester I found myself particularly stressed from the amount of work that I had to do and the fact that some of the work I had to do was writing and researching assignments which I generally do not enjoy doing, I find them very difficult and boring to do.
\\ 
The main factor to my stress was caused by the lack of time in which I had left myself to work on these assignments, if I had planned out the work load better I'm sure I wouldn't have been as stressed, as I was, and I think the quality of my work would of been a lot better if I had more time to work on the assignments. I'm sure this could of been easily avoided if I had started my research earlier and specifically planned out time to write drafts. 
\\
A SMART action I could of done to better manage the situation would of been as soon as I learnt about the assignment and figured out whether I think I would have trouble or not with it (like if it's something new to me that I'm not familiar with or if it's something I do not enjoy doing) would be to mark down time periods on my planner/calender to work on these assignments. 

\section{Second Key Skill: Talking about and planning work with others}

In week 6 I worked with my team on Vision in the dark during a studio practice week. While working with my team I felt that there wasn't much team communication and planning. I found myself having to ask each team member what they were working on in order to understand where the project was going and what I should be working on.
\\
Working with a team is still very new to me so it was quite challenging to do, generally I think the project ended up pretty well despite some miscommunication between certain people. There is still lots of room for improving my social and communicative skills amongst my team members. And as I have recently joined a bigger team of 10 people now I feel that communication and pre-planning work with them is going to be crucial for the quality of the project. Especially when planning the programming work with the 3 other programmers.
\\
A SMART action I could do to improve my skills would be to talk more regularly with my team about the direction in which the project is going and keep up to date with everything that everyone is working on and give my feedback when asked for or when I think it might be needed.

\section{Third Key Skill: Managing my own workload}

This skill kind of ties in with my first skill as like I mentioned I did not handle my workload very well during week 6 and 7. Since it was the first semester, I didn't really know what to expect from the assignments and the amount of work we were going to be given. Overall I think I did a fairly good job on getting my work done, but how I handle it needs to be improved.
\\
I often underestimate how long an assignment will take for me to finish so I should try and allocate more work time to each assignment I get especially if I think it's going to be a difficult assignment to do.
I also don't want to fall behind on work that might be needed for others to continue their work, like the team game project. I don't want to keep others waiting for me to finish a feature of the game or something.
\\
A SMART action I should do in the future is as soon as I learn about an assignment or task is to decide on what days I will work on it with my online calender.

\section{Fourth Key Skill: Self-learning}

In order to become a better programmer and games developer I need to be constantly learning new things. During this first semester I have learnt a lot from the timetabled workshops and lessons, but I have not been doing enough self directed learning. Self directed learning on specific subjects that could help me with projects that I am currently working on will help me a lot in terms of getting better quality work on feel less stuck when I have no idea what I'm supposed to be doing.
\\
For the team game project, I'm going to need to be doing a lot of Self directed learning as I'm very new to C++ and Unreal and even though we will be learning these within timetabled workshops, I don't think it will be enough.
\\
A SMART action I will do to improve self-learning will be to plan out a few hours on specific days like the weekends in which I will learn about a new topic that could help me with my current assignments/projects. I will mark these hours on my timetable.

\section{Fifth Key Skill: Applying my knowledge to current projects}

Applying my knowledge from the recent things that I have learnt has been tricky at times during this semester, but it has really worked out in other times, like when I managed to apply a lot of what I learnt from making a random map generator from the tinkering graphics project to the vision in the dark game demo. Apply what you learn is really important as it confirms that you know enough about the topic to use it successfully.
\\
This coming semester I'm going to need to try apply what i have learnt more regularly and it doesn't always need to be with a current assignment, I could instead try and apply what I learn to smaller personal projects to help me better understand what it is I am learning.
\\
A SMART action to help me with this would be after every new thing I learn that I think could help me in the future, would be to try and apply the knowledge to either a small personal project or a current assignment. Doing this will further fortify my learning and knowledge of the subject and help me become a better programmer and games developer.

\section{Conclusion}

In conclusion I want to focus my personal development on working more effectively by myself and with others and learn as much new and useful knowledge as possible in the field of computing and games development to further help me in becoming a professional programmer and games developer.
\\
I hope that doing these things will also help towards my future assignments and projects and help me become a better team member within team projects.
\\
This coming semester I will work towards my SMART actions/goals and make notes of my progress in my continued weekly reports.

\bibliographystyle{ieeetran}
\bibliography{references}

\end{document}
