% Please do not change the document class
\documentclass{scrartcl}

% Please do not change these packages
\usepackage[hidelinks]{hyperref}
\usepackage[none]{hyphenat}
\usepackage{setspace}
\doublespace

% You may add additional packages here
\usepackage{amsmath}
\usepackage{graphicx}

% Please include a clear, concise, and descriptive title
\title{CPD Episode 2: Revenge of the Miscommunication}

% Please do not change the subtitle
\subtitle{COMP130 - CPD Report}

% Please put your student number in the author field
\author{1707981}

\begin{document}

\maketitle

\section{Introduction} % 81 words
My ambitions remain the same. I'd love to create and run a company someday, standing in a high creative (yet also technical) position. However, coordinating people in such a way that doesn't confuse, frustrate or create co-dependency, will be more challenging than I realised. Who knew being a big leader guy would be hard!

Here lies the continuation of my constant adaptation to the social world in pursuit of the destruction of my social awkwardness and/or dumb decisions.

SUMMARY: What do I wanna do!
Lead a game company as a creative director! Be big idea guy who makes magic happen with his programming skillz!

How am I gonna get there!
Get better at communicating ideas!
- Use illustrative techniques!
- Get better at drawing!
Get better at programming!
- Make projects!
Get better at reaching out!
- Use more social medias!
Motivate other people!
- 
Get better at initiative!
Be a leader!


\section{Skills to improve}
\subsection{Affective - Regulating work stress} % 203 words (Woo!)
During PASS tutoring training, I teamed up with a peer to plan a session. We were very brief: about 15 minutes. We planned to follow-up on Facebook, but her messages came in 30-minute intervals, and were short--occasionally one-word answers. I was ultimately stressed in anticipation of a session where I didn't know what would happen, or what role I would play.

A fear of the unknown is a core part of my personality, with a bad effect on my health, yet a good influence on my work ethic. The stress, however, is unfavourable for myself and my peers. And ultimately, bad communication is unavoidable in the career, along with other circumstances wherein I can't predict the outcome.

Stress is broad, and various small things can reduce it. As a reputable fact for most people and a viable theory for programmers, exercise is known to improve mood at a biological level. So from the start of next term, I'll either a) run the Full Body Challenge on my Home Workout App or b) walk a lap around campus every Sunday morning. This will introduce more energetic and beneficial forms of break times into my routine, and may help me focus.

Specific: Yes
Measurable: No--how can I tell if my stress levels are reduced?
Achievable: Yes
Relevant: maybe?
Time-constrained: Yes

\subsection{Interpersonal - Taking initiative and motivating people} % 187 words (Boo!)
I was scrum master in Team Duo. To arrange stand-up times, I tried to determine the times where the most members would actually come in. It involved polls: ``what times are you free?'' and ``do you prefer post-session or pre-session standups?''. At the time, I felt that working around their believed ideals would make them comfortable coming in. But over time, I realised that others' ideals aren't necessarily oriented for good productivity.

Also, by essentially giving them decision, this may have created codependency, possibly damaging the dynamics of a scrum master and their team. I should have had more initiative, pushing for consistent times, observing how it worked, and adjusting it based on the results. Furthermore, the time doesn't matter so much as the motivation and enthusiasm, which should have taken focus.

My persuasion skills are lacking, and perhaps my people-reading skills need work too--both vital to motivate. One reputable book for these broad skills is \textit{``The 7 habits of highly effective people''} by Stephen Covey, which could help me learn actionable skills to read the dynamics and motivate the team. I'll read 6 pages per evening, with two days' leeway per week, with the goal of reaching the end of the book.

Specific: Yes
Measurable: Yes
Achievable: Yes
Relevant: The word count tho
Time-constrained: Yes

NOTICE: should I instead make a goal to e.g. criticise 5 people? no that doesn't sound so good... praise 5 people while providing criticism? that could work....but I need 5 people...use two members of the Italian mafia to threaten 5 people, with £500 profits per week?

\subsection{Dispositional - Self-prioritisation} % 198 words
Throughout the semester, my assignments took precedence over challenging real-life domestic tasks. I tend to focus on the goals of others. When I had time to work on my Arduino project or the group game project, I focused heavily on completing the latter. This was driven partly by a fear of submitting an incomplete-seeming game where the mark was shared across my team.

Judging by my actions, my motivation mostly comes from the expectation of meeting specific criteria--and I prefer letting others define it. When a criteria is vague, I lose motivation to meet them as i can't tell when it'll be done. Social anxiety can worsen the impact. 

To occupy myself over the summer, I aim to find a programming internship position somewhere in Brighton--another challenging prospect, that could benefit from being given specific criteria. This time using the LifeRPG app to track my progress, and maybe gaining motivation using its exp reward system, I will a) add my projects, music, artwork, and CV to my website, by Sunday 6/5/2018; b) update my LinkedIn profile by Wednesday 9/5/2018; and c) prospectively email 3 companies by Sunday 13/5/18.

======
NOTICE: This domain requires me to improve my dispositional skills. I.e. to manage my time in a diferent specific way, but isn't that always changing? Any technique I find, I'll likely find something better in a short time. What is the timeframe for these actionable goals?

Specific: Yes
Measurable: Yes
Achievable: Yes
Relevant: maybe not
Time-constrained: Yes

\subsection{Cognitive - Playtesting} % 201 words (fuuuu)
While acting as scrum master in Team Duo, there were many gaps in the agile process we worked on. A lack of Trello board updates required me to remind everyone, individually and regularly--this helped. A lack of understanding of attendance times led me to make a \#scrum channel on Slack for announcements--this increased the awareness. A preference by others led me to run stand-ups at the end of sessions--to negative effect--that I eventually reverted--to uncertain effect.

In Week 10, while playing the game, I realised what we'd been missing: playtests. Granted, getting the team to come in at unscheduled times required social wizardry and persuasiveness that I don't have yet. But I realised that playing the game is half the fun of making it. Many of the team somehow weren't doing this, and I could've made a greater effort to coordinate times for it to happen.

Next time, I'll make sure the team comes in at a specific, regular time once weekly for playtesting. This is highly circumstantial, but for now, let's say Monday afternoons, for half an hour, before the dedicated studio practice time. It might help motivate everyone to come in on time!

Specific: Yes
Measurable: Yes
Achievable: Yes
Relevant: Yes
Time-constrained: Yes

\subsection{Procedural - Recognising Order of Creation} % ???
The features I created in Handzer came in roughly this order: Images, arduino interfacing, debugging graphs, player, level, level editor, sprites, collision, lasers, bottles, smashing, animations, googly eyes, control polish, and enemies.

The resulting prototype had a few smash-able bottles with an enemy type, and fairly refined controls in a small level. This was disappointing, but I feel that my iterative approach was better than last time. My collision system, for example, started off as a flat rectangle check, even though I would have liked rotate-able level backgrounds. Furthermore, I created everything only once it was needed: string tools, arrays, and image caching.

Some features were still somewhat oversized for their purpose. The biggest time-eater, naturally, was the level editor, which saw little use in the end. An alternative approach would have been to make a few layers of backgrounds in Photoshop, and hard-code them in.

Identifying the most important features is hard. What's a good system for estimating this? Perhaps I could exploit the Pareto principle. When I next start a project, I will list all required features in a document, and split them into a top 20\% and bottom 80\%. To create a feedback loop, I'll do this style of retrospective each week, watching what worked or whether it went overboard. Behold, I have now created a new workflow and I will call it Agile.

Specific: Yes
Measurable: No
Achievable: Yes
Relevant: Yes
Time-constrained: No



\section{Recording Progress with the Diary} % 20 words


\section{Conclusion} % 93 words


\bibliographystyle{ieeetran}
\bibliography{references}

\newpage
\section{Appendix - Diary Template}
\begin{figure}[h]
\centering
\caption{Provisional CPD diary template}
\end{figure}

\end{document}
