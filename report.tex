% Please do not change the document class
\documentclass{scrartcl}

% Please do not change these packages
\usepackage[hidelinks]{hyperref}
\usepackage[none]{hyphenat}
\usepackage{setspace}
\doublespace

% You may add additional packages here
\usepackage{amsmath}
\usepackage{graphicx}

% Please include a clear, concise, and descriptive title
\title{CPD Episode 2: Revenge of the Miscommunication}

% Please do not change the subtitle
\subtitle{COMP130 - CPD Report}

% Please put your student number in the author field
\author{1707981}

\begin{document}

\maketitle

\section{Introduction} % 81 words
My ambitions remain the same. I'd love to create and run a company someday, standing in a high creative (yet also technical) position. However, coordinating people in such a way that doesn't confuse, frustrate or worse, create co-dependency, will be more challenging than I realised. Who knew being a big leader guy would be hard!

Here lies the continuation of my constant adaptation to the social world in pursuit of the destruction of my social awkwardness and/or dumb decisions.

\section{Skills to improve}
\subsection{Affective - Exercise and Become Big Und Stronk} % 194 words (Woo!)
When a project isn't going as well as I think it could, I get frustrated. During PASS tutoring training, I was teamed up with a peer to plan a session. We were much briefer than I wanted: about 15 minutes. We planned to follow-up on Facebook, but her messages came in 30-minute intervals, and were sometimes more than one-word answers. I was ultimately stressed by the bareness of the session plan, probably due to fear of the unknown.

I recognise that this stress happens whenever my (or my team's) work is unsatisfactory. This is a core part of my personality, and a bad effect on my health, yet a good influence on my work ethic.

Stress, however, is unfavourable for myself and my peers. To reduce work stress overall, I plan to take more breaks with more exercise. As a reputable fact for most people and a viable theory for programmers, exercise is known to improve mood. And buns and thighs. So from the start of next term, I'll either a) run the Full Body Challenge on my Home Workout App or b) take a lap around campus every Sunday morning.

Specific: Yes
Measurable: Yes
Achievable: Yes
Relevant: Yes
Time-constrained: Yes

\subsection{Interpersonal - Persuasion} % 210 words (Boo!)
I was scrum master in Team Duo. To arrange stand-up times, I tried to gauge the times that the team would come in. It involved polls: ``what times are you free?'' and ``do you prefer post-session or pre-session standups?''. At the time, I felt that pertaining to the interests of others would motivate them the most. But over time, I realised that others' interests didn't always pertain to good productivity. It's nature: given the decision, many would prefer to `work' from home. Yet I was practically giving the decision to them.

This codependent approach probably damaged the natural dynamics of a scrum master and their team. I should have approached stand-up times like a game testing session: Make the game, then observe how they actually play it, rather than ask what game they want.

Were I more skilled and confident, I would have taken that above approach from the beginning. However my aptitude in reading and persuading a team is lacking. To improve my skills recognising a team's dynamics, and putting myself in a more correct position when acting as scrum master, I'm reading 100 pages of ``Influence'' by Robert Cialdini, at 8 pages per evening. This could improve my outreach skills and anxiety as well.

Specific: Yes
Measurable: Yes
Achievable: Yes
Relevant: Yes
Time-constrained: Yes

\subsection{Dispositional - } % 203 words (oh dear)
Throughout the week, 


\subsection{Cognitive} % 203 words (oh dear oh dear)


\subsection{Procedural} % 190 words (what a save!)
In 


\section{Recording Progress with the Diary} % 20 words


\section{Conclusion} % 93 words


\bibliographystyle{ieeetran}
\bibliography{references}

\newpage
\section{Appendix - Diary Template}
\begin{figure}[h]
\centering
\caption{Provisional CPD diary template}
\end{figure}

\end{document}
