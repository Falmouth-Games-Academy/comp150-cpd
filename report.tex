% Please do not change the document class
\documentclass{scrartcl}

% Please do not change these packages
\usepackage[hidelinks]{hyperref}
\usepackage[none]{hyphenat}
\usepackage{setspace}
\doublespace

% You may add additional packages here
\usepackage{amsmath}

% Please include a clear, concise, and descriptive title
\title{First semester - Progress, challenges, actions}

% Please do not change the subtitle
\subtitle{COMP150 - CPD Report}

% Please put your student number in the author field
\author{1607934}

\begin{document}

\maketitle

\section{Introduction}

Currently, it is uncertain which career path/goal I intend to choose, however I am confident of it being in the games industry. I am interested in continuously developing my programming knowledge, as well as learning the Agile ways and the games industry in general. Challenges I have encountered are as follow: understanding code structure, group work engagement, independent work research, using Git systems such as branching and pull requests, and presentation skills. 

\section{Understanding code structure}
One of the major things that has been both apparent and challenging for me is understanding code structure. This is obviously a vital skill to have, as it covers everything to do with computing. I am able to understand simple code processes like the ones done during the tinkering-graphics lectures, although anything further than that has been completely incomprehensible to me. Nothing sticks, even when everything has been explained in great detail. As expected, this overwhelmingly affected my performance and overall motivation when I worked in teams where I didn't have much to offer at all, forcing my colleague(s) to do most of the work, which I found awfully discouraging and disappointing. Even more so since I see myself as a committed person. To address this, I have decided to practice more during the break in order to catch up, especially since we will be working with the BA's next year. Thereafter, I'd like to compare the results with the current lines of code I have, which should indicate my progress of learning to understand code structure or programming in general.  

\section{Group work engagement}
The second skill goes in conjunction with the first skill. Especially in the games industry, it is important to feel comfortable and to provide personal thoughts within a group, as they establish structure and form a communicative circle which help group members remain on the same page. I struggled with this both due to my understanding of code structure as mentioned, which currently isn't much, and due to my overall confidence. I have had multiple ideas and opinions during group projects, although I end up keeping most of them to myself as a result of my lacking knowledge for the specific topics we worked with - even though they were thoroughly explained. To overcome this, I will try to push myself more and communicate more with group members in order to improve my engagement and prove my commitment. I will then analyse if my overall confidence within groups has increased overtime. 

\section{Independent work research}
Unsurprisingly, finding the appropriate sources for my Agile essay proved to be a challenge. While the importance of academic research in the games industry can be argued, it can prove useful to acknowledge and understand the various methodologies, approaches and general opinions that could be potentially applied - or at least considered - to oneself or in a workplace. As for me, I had to search and change between multiple sources that would fit my research question. While this is not unusual, I spent too much time on this instead of refining what I actually had. This affected the final version of my essay, as I ended up underutilising the research sources and at times wrote vague assumptions. For next time, I will try preparing and searching online ahead of time in contrast to doing everything a few weeks prior the deadline. The quality of my second essay should be an indication of how much I have improved in terms of independent work research. Comparing my old and newer essays should be considered.

\section{Using Git systems}
In the games industry and the like, learning to use version control systems such as Github, among others, is key in order to communicate and understand different aspects of a group's work. Examples of these include branching and pull requests. During our work with various groups, I have not been utilising the branching system to its potential. Rather than creating multiple versions of the same file in the master branch (which was what I primarily did), it would've been easier and cleaner to create test branches in order to avoid mixing up important files that should not be mixed. Additionally, my use of pull requests could be improved. While I did use them, they could've been used more frequently and earlier rather than at the last minute. For these reasons, I will focus on utilising the branching system as it was meant to be used, as well as trying to use pull requests more in order to receive feedback from my peers to further improve my skills. 

\section{Presentation skills}
Presenting and pitching to an audience are skills that should not be taken lightly, as they have the ability to prove one's engagement and commitment in a project, which also increases the chances of receiving investment from investors. Throughout our diverse course work, we have had to present and pitch to different audiences with varying sets of backgrounds. After doing these, I later realised that I need improvement in speaking more confidently and clearly. As a result, I have decided to devote time in practicing my presenting skills by simply reading off a script, then shifting to a more independent stance based on it. Another idea I found interesting was to record myself while presenting (either video or audio), then analyse from there what needs to be improved. 

\section{Conclusion}
Overall, I think accomplishing the actions above will prove beneficial in terms of my continuous development both as a person and computing/academic expert. Acknowledging, analysing, and improving my actions will aid myself and potentially my peers in becoming better accustomed to the development industry, subsequently increasing chances of employability. In addition, these should help towards my future university work, decreasing possible problems that could hinder my learning experience.   

\end{document}
