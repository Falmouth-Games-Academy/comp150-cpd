% Please do not change the document class
\documentclass{scrartcl}

% Please do not change these packages
\usepackage[hidelinks]{hyperref}
\usepackage[none]{hyphenat}
\usepackage{setspace}
\doublespace

% You may add additional packages here
\usepackage{amsmath}

% Please include a clear, concise, and descriptive title
\title{Improvement by Review}

% Please do not change the subtitle
\subtitle{COMP150 - CPD Report}

% Please put your student number in the author field
\author{1706165}

\begin{document}

\maketitle

\section{Introduction}

This report will entail a review of 5 specific key skills that I have gathered throughout the weekly reports as well as my potential aspirations in the industry. My career goal within this industry is to be able to create a unique game that changes the standards for modern gaming as a whole in a positive way, the same way that "Super Mario Bros" changed the landscape and future of digital games, I wish to do the same with a game of my own and set an example for generations to follow. The five skills that I will be reviewing in this report include adaptability, concise communication, programming proficiency, research and presentation skills. 

\section{Adaptability}

During my time within the University, I have found myself in unfamiliar situations that I have never really dealt with before and such, have over-stressed myself over the outcome and how I could have impacted it differently via a careful and thought through course of action instead of uncontrollable panic and improvisation. Similarly, my experience with the presentation of the Game Project was a learning tool as our game was not complete and the presentation itself was mostly improvised by the group. This is an instance where I had to adapt to the situation and do the best that I could given the circumstances. Within the industry, if my adaptability within the industry is slacking as a leader of a project, it may reflect on my colleagues and affect their mental state as disarray and unknown cause panic at troubling times. I could apply a sense of confidence and awareness in challenging situations to prevent complete break-down. This skill in particular could be improved by setting tasks such as reading a book, exercising and such, each day and complete them daily to gain self-assurance and confidence within oneself over a period of time. Furthermore, adaptability may come as a result of many years of industry experience and tackling issues on a day-to-day basis with problem solving.

\section{Concise Communication}

Up to this point, Concise Communication has been my main issue as I am guilty of not saying enough and missing the point when working within a group setting. When discussing ideas and potential features of a game in the works, I tend to skip out on my own ideas and reflection of other people's ideas in favour of their opinion over my own. This defeats the purpose of a discussion and I am purposely skipping out on my own opinions which may affect the project in order to satisfy and not be scrumptious towards my other group members. This may stem from my discontent towards discussions as a whole starting from early school days where my opinion wasn't really valued, I have been working on altering this way of thought and have found some success but sometimes, it tends to slip away. In addition, patterns of passive aggressiveness towards my group members when they haven't been attending is a negative pattern that I have developed as a way to deal with my frustrations towards the other members. To fix this, I should speak with clear intent and during meetings I should raise at least two points per meeting about either my own idea or a review of someone else's over a period of 3 months to make sure it is engrained.


\section{Programming Proficiency}

This skill in particular is a crucial skill to nail as I have found myself looking at the work of my colleagues and having minimum understanding towards their goal with the code without consultation beforehand. Reaching a point where I can safely look at the code and understand its purpose by my own expertise and understanding rather than a review from the developer. Although, consultation and asking questions can lead to picking up some of the techniques that others have used in their code and may increase my understanding as a result. In addition to understanding, it is also helpful to give recommendation if a group member is struggling which is something that I haven't been able to do so far due to my lack of understanding in some areas. In truth, this lack of understanding has intimidated me from taking more difficult programming tasks which is something that I should be doing to improve. Finally, to improve this skill I can go read programming documentation or follow educational You-tube tutorials once a day over a period of 5 months, this will assure that I am normalizing looking at code and making it easily de- constructable in the future. This way of practice in particular will assure that I can help others whilst at the same time improving my proficiency.  

\section{Research}

Throughout the study block, I have encountered times within the sessions that there is a certain topic or technique which is linked to the session that I have no knowledge about which in turn, creates a missing slot that could be linked to a more in-depth understanding of the topic. More outside research would make sure that I am constantly in the loop and can follow along within the session without feeling hopeless. In addition, I have encountered the same issue within my programming as more often that not, there is a more efficient way of doing something that is better than my own. Having the tendency to keep my own written code and ignoring a more efficient way, this is hurting my own practice as experimentation with easier techniques will lessen the time I spend on a program with faster methods. Researching different techniques to do the same thing will give an alternative and arguably better results in the long run. I could fix this by writing down the topics or words that I cannot fathom and then researching in my own time, doing this throughout the week will expand my knowledge overall and will open up more topics of conversation.

\section{Public Speaking}
Presentations themselves have never been my strong suit throughout my time in education. Anxiousness before and nervousness during a presentation is something that I have to deal with each time. Although may be not noticeable but the pressure is sometimes daunting on me and I tend to try and get through everything as fast as possible, this is reflected in my speech as my main goal is to be done with it as soon as possible. I have done multiple presentations up to this point and have improved in terms of being able to get it done whereas before I would just crumble. Being comfortable with speaking in front of people will greatly improve my pitches of games and feeling relaxed when doing a presentation will assure that I am still getting through the content but doing it at my own pace without fast forwarding. Leading a group of 10+ people, if I am not confident in public speaking and unclear with my points, it could scrutinize the whole project as everyone would follow their own perspective if my ideas are misunderstood. To fix this, I need to prepare for a presentation properly via written cards and make sure to prepare content to then present as well as constant practice in front of other people to make it seem less intimidating.


\section{Conclusion}
In conclusion, having the extra practice at being Adaptable will help deal with any situation that is encountered and having the confidence to deal with it will be beneficial to the team in the long run as well as proper discussions with no input ignored will aid in error avoidance. Being able to research programming topics and misunderstood lesson topics is crucial to expanding understanding of linked subjects. Furthermore, having a successful pitch via public speaking improvement may impact the game's success and funding as well as being linked to inter-personal skills such as confidence or team-working.

\bibliographystyle{ieeetran}
\bibliography{references}

\end{document}
