\documentclass[11pt]{article}

\begin{document}
\title{CPD Tasks}
{\section*{Week 0}
{\small\textbf{Induction Week and the Robot Olympiad}}\\
The Lego robot Olympics was a fun and challenging event, it really brought out my problem solving and collaborative skills as I had to work in a team and together we needed to  construct a robot from Lego pieces while making it as functional as possible. While building the robot we had to iterate and modify it a lot, as it needed to change and adapt to the different space challenges it was put through. What was really challenging to us as a team was understanding how to use the programming software to program the Lego NXT robot, none of us had much prior knowledge on how to use it, somehow we still managed to get some of our programs to work according to plan thanks to the bits we did understand.
 
}
{\section*{Week 1}
{\small \textbf{From Concepts to Design and Commerce, An Introduction to Python Programming and PyGame, and Computing Fundamentals}}\\
This week's COMP150 tasks was an interesting one, designing a game in a team and creating a pitch to sell the idea to investors was totally out of my comfort zone, this was my first real time developing a game as a team and it worked surprisingly well, most of the team members had a good idea of the game we wanted to design and we were all collaborating together to come up with a cool game concept. What we found most challenging was coming up with a target market that would be interested in buying our game, sales numbers and how much we would need from the investors in order to potentially develop this game. The whole experience was definitely a huge learning step for me.\\
In COMP120 we learnt the basics of Python, I was happy to finally get started to learn a new programming language. During the week I encountered quite a few programming issues while doing some of the COMP120 activities, simply talking to my peers helped me get past those problems and better understand where I was going wrong.
}
\\~\\
\\

{\section*{Week 2}
{\small \textbf{Tinkering Graphics, the Agile Philosophy, and Journal Articles in Computing}}\\
In COMP120 we have been learning the basics of how images and computer graphics are displayed on screen and how they can be modified with code. We had some activities to do in pair programming that were quite difficult as we were kind of thrown right into it, the whole class seemed quite agitated by the difficulties of not knowing enough about python and pygame to do the activities, a lot of the documentation was quite vague about how to go about using certain functions in order to modify pixels. In our own time my pair programmer and I got online together to continue working on the Python/Pygame activities, and together we actually started to better understand how it all worked.\\
In COMP150 we have been given the task to come up a with a 2D experimental game idea and pitch it to the rest of my peers, so far i have been finding the task to create a presentation alone really difficult, and the idea of  in front of a large audience alone is out of my comfort zone, put I'm up for the challenge and positive it will go well.
}


{\section*{Week 3}
{\small \textbf{Starting my Tinkering Graphics Assignment, Pitching my First Game Concept, and Version Control}}\\
This week in COMP150 I did my game pitch and apart from the fact that the fire alarm went off while I was presenting, I think my presentation went well, of course I still need to improve on my public speaking skills, but I'm getter there. Unfortunately my game idea didn't get enough funding to go into development but it was close, instead I will be working in a team of 4 on Echo in the dark. So far as a team we have reworked the game idea and created a games design document, we are now getting ready to start development on the game and get the base mechanics and ideas down.\\
In COMP120 we have been given code re-purposing contract assignment to do, my pair programmer and I have decided to do the map generation contract in order to learn a bit more about how maps could be procedurally generated in games and in turn this would help us create he procedurally generated maps for our COMP150 game.
}

{\section*{Week 4}
{\small \textbf{Working on Assignment1 - Code Repurposing I (Tinkering Graphics)}}\\
This week I have been working a lot on the Code Re purposing Assignment, I have been finding it really challenging to do as I'm still quite unfamiliar with Python and the way it works, but I'm slowly getting there and I can see the progress happening. So far my map generator is working but it's rules for how the tile paths align still needs fixing for it to work properly. Hopefully I will have it working for next weeks deadline.
}

{\section*{Week 5}
{\small \textbf{finishing Assignment1 - Code Repurposing and working on VisionInTheDark}}\\
I finally got the map generator working exactly like I wanted it to, and I feel great for finally managing to get past each hurdle to get it working as intended. With my pair programmer we cleaned up the code and submitted the work for the deadline. Spent Wednesday and Thursday working with my team on the game demo project, so far it's coming along slowly, we are still trying to figure out how the main player movement will function but we are getting there.
}

{\section*{Week 6}
{\small \textbf{Studio Practice week working on the game}}\\
This week was spent entirely in the games studio working with my team on the vision is the dark game demo. It was a difficult and challenging week but overall it's been very fun. Our sprint goal was to have a basic movement, vision and collision systems with a simple map generator which we have managed to do which is great, I focused on creating the map generator as I had the most experience with it from doing the code re purposing assignment, I also did some work on the movement system and fixing the collisions so they work alongside the map generator and helped refactor code and simplify some stuff. working with the team did get a little difficult at times as not everyone really knew what to work on, but we managed to all do something to contribute.
}
\\
{\section*{Week 7}
{\small \textbf{Learning about sound and writing the essay}}\\
This week we learnt about the basics of sound and how it's created digitally using python. I didn't quite understand all of it but I think I'm getting there. And I also spent quite a bit of time working on the essay which i found extremely difficult to write and did not enjoy doing at all. Managing my time between all the different assignments has been quite difficult this week and I expect it to be even harder next week.
}

{\section*{Week 8}
{\small \textbf{Finishing the essay and writing research journals}}\\
This week was very chaotic for me, from doing the peer-reviews for the essay to finishing it and then doing the research journal and doing the peer-review for that and finishing it, I did not have a good week. I do not enjoy doing any kind of academical writing type assignments, it's just not my thing. I did enjoy reading and researching for my research journal but writing was a bore and very frustrating to write, but the essay was probably worse as researching agile philosophy was quite boring.
}

{\section*{Week 9}
{\small \textbf{Finishing up the team game project and working some more on the tinkering sound assignment}}\\
This week was mainly spent working on the team project and making sure it was worthy of being demonstrated as a demo for the team game presentations on the next Monday, we added some new small features and fixed a few bugs, we also had to prepare a presentation which wasn't very fun to do. Apart from the team project I also needed to do some more work on the music generator with my pair programmer for the COMP 120 assignment, so far the project has been going pretty well but the deadline is closing up and we aren't quite where we want to be with it, so hopefully we can work it out a bit more for the deadline.
}
\\~\\
{\section*{Week 10}
{\small \textbf{Finishing the audio project and comment up with game ideas for the new team game project}}\\
This week I met my new team with the BA's and we came up with many different concepts for the new game we will be working on after Christmas, So far we have a basic idea of a top down multi player couch versus shooter,  I'm not sure how feasible this idea is to build in Unreal but I'm sure I will find out soon when I start to properly learn to use Unreal. My pair programmer and I managed to finish up the tinkering audio project, I'm quite happy with what we have managed to do for this assignment, especially for the fact that I didn't know anything about how audio worked in computing, I thought that it was very interesting to learn about and I'm happy with what we have accomplished.
}

{\section*{Week 11}
{\small \textbf{Learning Unreal and doing worksheet D}}\\
This week we finally started to learn Unreal which makes me really happy. I don't have any previous experience with Unreal but I have always wanted it explore it and try and make a game with it. Learning Unreal is a little daunting as it's quite a bit more complex than Unity which I am familiar with and I will also be learning C++ which I have heard is a bit of a pain to learn, but I'm up for the challenge and can't wait to learn more! This week I also did worksheet D which was fun to do, In work sheet D I needed to create function in order to make a noughts and crosses game work, I did find it tricky to implement the function that checks all the tiles but other than that the exercise was fun to do.
}

{\section*{Week 12}
{\small \textbf{Team pitch and quiz D}}\\
Seeing as this week was the last week before the winter holiday break not much happened. The main thing that needed to be done was pitch the team game idea to some teachers. Preparing the pitch was a little stressful at times but overall it went pretty well and performing the pitch went generally quite well, the teachers didn't have many negatives things to say about the idea so that's good. On Friday we reviewed the maths quiz D which was about rotating points in space, it took me a while to understand and do this quiz, but I managed in the end and I feel that I have a better understanding of the subject now.
}

\end{document}