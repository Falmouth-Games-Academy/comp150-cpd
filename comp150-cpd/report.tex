% Please do not change the document class
\documentclass{scrartcl}

% Please do not change these packages
\usepackage[hidelinks]{hyperref}
\usepackage[none]{hyphenat}
\usepackage{setspace}
\doublespace

% You may add additional packages here
\usepackage{amsmath}

% Please include a clear, concise, and descriptive title
\title{COMP150 - CPD Report}

% Please do not change the subtitle
\subtitle{COMP150 - CPD Report}

% Please put your student number in the author field
\author{1606119}

\begin{document}

\maketitle

\section{Introduction}

With my career goal lying in creating code for the physics engines in games, there are some key skills that I feel are important for me to improve on to make me a better fit a development position, these being: meshing with a new team, the learning of new development tools or coding languages, presentation skills, time management on coding projects and use of version control and branching. And hopefully by improving on these skills, I can hopefully overcome the obstacles and improve the quality of my future work, both in education and future employment roles.

\section{Meshing with new group}

The first skill that came to my attention when looking back at my CPD reports was meshing with a new group of people/team, I have always struggled with this as I don't find it easy to talk to new people, which in the past has lead to myself not speaking my mind about a group decision and getting stuck in a project I didn't have any drive to complete and such feel behind on the work and got a lower grade than I'd hoped. Obviously this skill would be important from a games development perspective as if I was put in a new team, I would not be able to effectively communicate with the other members, which could lead to me being mis-assigned work or my feedback not being implemented into the product, which could degrade from the quality of the final piece as it has in my past projects. I will overcome this obstacle by pushing myself out of my comfort zone when I am put in a new team and lay down my areas of expertise and what I am comfortable undertaking, with progress being marked by myself feeling more confident in my work and the quality of the final product. 


\section{The learning of new development tools or coding language}

The second skill that came to my attention was the learning of new development tools or coding languages efficiently, the relevance of this skill is obvious as if I enter a new job scenario, where they use tools or languages that I've never used before, I will spend the majority of my time trying to work them, which will cut into my actual development time, so this skill is very important to learn before I enter a scenario like that.  As stated before, if I can't use the tools and languages in question, which I currently struggle with as it takes longer than it should for me to learn my way around new software, as I'm too worried about using all of the features well, instead of the ones I need, the quality of my work will be affected as I spend more time learning it than actually working. To overcome this I plan to going forward, look at what languages and tools I'll be using in future projects and learn them before hand, which help me get right into the workflow and help develop with it being measurable by how little reference to the documentation I have to do. 


\section{Presentation Skills}

The next key skill that came to my attention was presentation skills, I find that if I have to present anything, I talk far too much about the small details and end up running over my time, because I think that a better knowledge of them allows a better bigger picture, but on reflection, they just seem to make it more confusing and add unnecessary length, normally before I reach important parts such as budget, and such to that, this would hinder my work in a professional context as I may be pitching a new feature or gameplay idea, and I'll run over the time limit or just bore the Product Owner by worrying about the minute details, so it is very important that I learn to alter my presentation style. The quality of my work is normally affected by my lack of skills as mentioned before is that I talk too much about small details, where one feature may have been ground breaking, and the person watching didn't get to see it because I went over the time limit, so it could have a large affect on my end product. To overcome this obstacle, I will talk about only what needs to be outlined, and let the PO ask questions if they want to know about anything in deeper detail when I've finished, with this being measurable by if my future presentations finish within the time limit. 


\section{Time management on coding projects}

My fourth key skill I need to work on is time management on coding projects, I find that I always underestimate the amount of time that a project will take and am left scrambling at the last minute to finish it, but due to the nature of code, it's near impossible to finish it in time to the standard that I wanted, and such the mark I receive for the work is lower than I wanted. This skill is very important in a professional context as if I leave work incomplete just before a milestone or deadline, my code may be filled with bugs and errors or may not work with my peers work, which could affect the development time of the whole project. To overcome this obstacle, I will evaluate a project on how long it will take, then spread the work out evenly, as too much at one time can cause overload and errors may appear, I will then have to work on my discipline and force myself to get the work done as not to fall behind on the deadlines, and if this plan is successful, I should have work completed before the deadline, which leaves me room to improve it. 



\section{Use of Version Control and Branching}

My final key skill is that of the use of version control and branching, I find that the code that I create or work on nearly always had to be fixed before I can push it as it will clash with my peers work, because on reflection, and as seen in my CPD tasks, I get too hung up the formatting of my code, and if it looks good, which takes up a lot of time and by then my peers have normally submitted and my code clashes because we were working on the same block or operation, which takes up more of the development time and may push our grade down due to that. This skill is important in a professional context, as mob or pair programming might not be possible and I might not know what others are working on and our work may end up clashing, which we would then have to fix and resubmit, which cuts into development time. To overcome this problem, I will take another look at the tutorials I was provided about utilising branches, so that different members of the team can work on separate pieces at different at a time, thus my code should clash a lot less if I can effectively utilise it. 

\section{Conclusion}

Write your conclusion here. Though the conclusion should be brief, no more than 100 words, it should do more than merely summarise the report. Focus on the five SMART actions that you intend to take in order to overcome any challenges and/or obstacles. Contextualise how this will help you towards your intended career goal and how this may improve your project for the next semester.

The five skills that I have outlined I think will help me improve my work in the future by allowing me to be much more productive, create a better end product and be a more active member of a team, with these skills all very important to successful development, which hopefully my future employer will be looking for, and by working on these skills now, I will have them mastered by the end of my educational period, and they will help in my future projects here, hopefully improving my contribution to the project next term. 

\bibliographystyle{ieeetran}
\bibliography{references}

\end{document}
