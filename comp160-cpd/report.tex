% Please do not change the document class
\documentclass{scrartcl}

% Please do not change these packages
\usepackage[hidelinks]{hyperref}
\usepackage[none]{hyphenat}
\usepackage{setspace}
\doublespace

% You may add additional packages here
\usepackage{amsmath}

% Please include a clear, concise, and descriptive title
\title{COMP160 - CPD Report}

% Please do not change the subtitle
\subtitle{COMP160 - CPD Report}

% Please put your student number in the author field
\author{1606119}

\begin{document}

\maketitle

\section{Introduction}



\section{Communicating effectively with team members about job allocation}
The first skill that came to my attention when looking back at my CPD reports was
meshing with a new group of people/team, I have always struggled with this as I don’t
find it easy to talk to new people, which in the past has lead to myself not speaking
my mind about a group decision and getting stuck in a project I didn’t have any drive
to complete and such feel behind on the work and got a lower grade than I’d hoped.
Obviously this skill would be important from a games development perspective as if I
was put in a new team, I would not be able to effectively communicate with the other
members, which could lead to me being mis-assigned work or my feedback not being
implemented into the product, which could degrade from the quality of the final piece
as it has in my past projects. I will overcome this obstacle by pushing myself out of
my comfort zone when I am put in a new team and lay down my areas of expertise and
what I am comfortable undertaking, with progress being marked by myself feeling more
confident in my work and the quality of the final product.

The first skill that came to my attention was that I do not communicate effectively with my team about my job allocation, and I have started to find myself undertaking work, only find out it has already been done, or a colleague is in the development process. 

\section{Use of version control and branching}
Learn to use all of the available features of the VC as not having full use of it lead my work causing conflictions and slowed develop time. 


\section{Learning new tools and skills}
I was nearly almost behind when learning new things such as using the server and blueprints. 

\section{Time management on coding projects}
Time managment is still plaguing me from last term, although I had started to use my calender to plot out when I had dates due 

\section{Finding solutions for new problems}
Never go outside of my comfort zone as I find it hard to ask for help and end up wasting my time. 

\section{Conclusion}

Write your conclusion here. Though the conclusion should be brief, no more than 100 words, it should do more than merely summarise the report. Focus on the five SMART actions that you intend to take in order to overcome any challenges and/or obstacles. Contextualise how this will help you towards your intended career goal and how this may improve your project for the next semester.



\bibliographystyle{ieeetran}
\bibliography{references}

\end{document}
