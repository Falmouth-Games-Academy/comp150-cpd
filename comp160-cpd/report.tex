% Please do not change the document class
\documentclass{scrartcl}

% Please do not change these packages
\usepackage[hidelinks]{hyperref}
\usepackage[none]{hyphenat}
\usepackage{setspace}
\doublespace

% You may add additional packages here
\usepackage{amsmath}

% Please include a clear, concise, and descriptive title
\title{COMP160 - CPD Report}

% Please do not change the subtitle
\subtitle{COMP160 - CPD Report}

% Please put your student number in the author field
\author{1606119}

\begin{document}

\maketitle

\section{Introduction}
Having worked on my highlighted skills from last semester over the duration of this semester, some of them I solved with relative ease by following the steps I laid out in the last report, however some I found were a lot harder to work on, which is why I have carried them over to this semester's report as I think these are skills that I need to ascertain to become successful in my future education or work. The new set of skills I have outlined in this report have come to light when working on the game project in collaboration with the BA's, and thus I feel that this set of skills will be important to have to effectively work in a group with other specialisations as this is very common practice in real world development. 


\section{Communicating effectively with team members}
The first skill that came to my attention was that I do not communicate effectively with my team about my job allocation, and I have started to find myself undertaking work, only find out it has already been done, or a colleague was currently is in the development process. Without this skill, my future work in a professional context would be very difficult as the development cycles of games are very finely planned, and my small piece of work might be integral to a much larger part being completed, so it is important that I gain this skill before moving forward, as without it, my work flow may hold up others, and then I may rush to get it done and produce poor work, which is not favourable in any context. To overcome this issue, I will endeavour to communicate more with teammates about the jobs at the hand and what I should be doing to make myself most useful, and if I do this, I should see an increase my productivity and more of my work getting into the project, and as I will start work on this task as soon as possible, I should see a change soon, or at least when I start my next group project; and I feel that this task is more than realistic as it only involves me communicating more, which is easily done. 

\section{Use of version control and branching}
The second key skill that I found I lacked was that of the mastery of new version control, having been presented with a new system in the form of SVN for use with our new group project, I struggled to get to grips with how it operated as it varies quite a lot of from GitHub, and I found that a lot of my commits would flag up as causing conflicts, which wasted a lot of time as I had to manually fix the errors, but on reflection I should have been pushing the changes to an experimental branch to make sure they worked. Without this skill, I can see that I could spend a lot of my time fixing commits where as I could be completing tasks, which in a professional context is very bad as my contribution to projects would be affected. To overcome this issue, I will go back over the SVN tutorials on Pluralsite over the summer break ,so that I can effectively make use of all of the VC's features and that I can confidently use branching so when I get back and have to start work on the 2nd year project I can be a more effective number of the team and not cause any problems when committing my work. 

\section{Learning new tools and skills}
The third key skill that I want to cover is the learning of new skills or tools, this became evident near the start of the term when we were first taught blueprints for unreal, and where as everyone else had a good grasp of them, I struggled to get to grips with them as quickly, this might have been down to my lack of drive to look at the online tutorials we were linked to, as on reflection I think I would have picked it up a whole lot quicker if I had studied the necessary tutorials. And as I will most likely be using a different engine next year, it's very important that I prioritise learning these new tools so that I can be productive and contribute effectively to projects I'll be taking part in, as this would also be expected in a professional context. To overcome this issue, I will assert myself to learn about all of the tools I will need to use for next year by looking at relevant documentation and start experimenting with the software, so I will be competent with it by the time that I return, and I can then work more effectively with other members of my team. 

\section{Time management on coding projects}
The fourth key skill I want to outline is time management on coding projects especially, and although I covered this in my last report, it still plagues me. I find that I worry more about what the aesthetics of a project over the actual code, and spend the majority of my time making it look nice rather than make it actually function, which normally leads to me rushing to finish the code near the due date, leaving it with sporadic commenting and sometimes not all of the features I wanted, which on reflection lead to me losing a fair few marks. To overcome this issue, I will endeavour to get the coding portions of projects done first before worrying about the aesthetic side completed, and I will make better use of my calendar and time keeping applications so that I know how much time I have left to complete a task or project, and can effectively plan out how much work I have to complete in a set amount of time so that nothing is left incomplete or not to the standard it should be. 

\section{Asking for help with new problems}
The last skill that I found I lacked was finding solutions to problems I haven't encountered before, this became especially evident quite recently with comp140 controller project, I had a lot of problems with my code at the start, and instead of asking for help, I simply stuck my head in the sand, it then got close to the deadline and my controller was still having problems, I then asked for help, where as I should have asked as soon as an issue developed, as my final product would have been of a much higher quality, so this is defiantly a skill I need to work on to ensure my quality of work goes up and problems with my work don't impede my ability to get work done. To overcome this issue I will have to work on my fear of approaching people for help obviously, I can do this by starting to make use of the office hours that my lecturers have so that I can discuss any issues I'm having before it puts me behind on my work, I can also make better use of Slack by getting in direct contact when a problem arises so that I can hopefully resolve the issue quickly. 

\section{Conclusion}
To conclude, the SMART actions I outlined in this report will hopefully assist me in producing much higher quality work, much quicker and will help my ability to work in a team more effectively, which will help massively with the new team project next year and hopefully my future proffesional career as they are all important to allow me to work effectively in a team and on my own. 

Please find me weekly reports in the comp150 repo.





\bibliographystyle{ieeetran}
\bibliography{references}

\end{document}
