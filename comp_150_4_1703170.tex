
% Please do not change the document class
\documentclass{scrartcl}

% Please do not change these packages
\usepackage[hidelinks]{hyperref}
\usepackage[none]{hyphenat}
\usepackage{setspace}
\doublespace

% You may add additional packages here
\usepackage{amsmath}

% Please include a clear, concise, and descriptive title
\title{Continuing Professional Development - Semester One}

% Please do not change the subtitle
\subtitle{COMP150 - CPD Report}

% Please put your student number in the author field
\author{1703170}

\begin{document}

\maketitle

\section{Introduction}

The overarching goal of my studies in computing for games is currently to become well versed enough in programming to one day be able to either create my own games from scratch on my own or within a small indie group as a developer. I think of myself as a creative person and would like to gain the technical skills and diligence required to become a professional game developer in the future. My five key skills I need to focus on are:

\begin{itemize}
  \item Stress Management
  \item Communicating Ideas In A Group
  \item Version Control Management
  \item Self Motivated Learning
  \item Logical Problem Solving
\end{itemize}

\section{Stress Management}

The majority of the stress I have felt has come from a lack of confidence in my own abilities to do the work well which makes me put it off until I have to do it. Then my motivation to finish something comes mainly from a fear of failure. This is a very negative way to respond in life in general and especially work life. This was most evident in weeks seven and eight in which we had to hand in two academic papers in short succession. Having not done any form of essay writing for at least ten years I felt very apprehensive when faced with this new challenge, so much so that I waited until the weekend before it needed to be handed in to even start them. I also felt very stressed when pitching in week three and ten. 
\\
My SMART goal to curtail this stressful environment that I place myself in at times would be this: for my next academic essay or writing, I will begin any research of the topic as soon we are given it. I will spend at least the equivalent of three hours a week on the research or essay until I have a draft and will complete it at least three days before the deadline. I also intend on working through my fears of speaking in pitches by rehearsing any presentations thoroughly at least two days before the pitch to ensure I am confident in what I am saying.

\section{Communicating Ideas In A Group}

Working in close knit teams has been new and challenging for me since week 0. I feel that my introvert nature is a hindrance to team communication somewhat, this was evidenced throughout our first group project. During the project the team very rarely communicated well as the project went forward and I felt unable and too uncomfortable to speak with team members at times where some constructive feedback may have been useful. Also, when I did make any design changes that another team member didn't like I would back down immediately rather than defend my position on why the change was made or seek out a compromise. This hesitation can prevent the game from improving. This was also an issue in my first pair programming assignment. I felt too anxious and perhaps too indecisive to speak up arrange to meet up and work on things if I didn't think enough work was being done on the task. 
\\
My SMART goal to combat this would be to make a conscious effort to provide a new idea on how to change or improve something in our game or team every sprint review in my new development team.


\section{Version Control Management}

During our project I would sometimes work on our game on my own at home and not knowing GitHub very well I would save different versions of the game to my local computer without committing it. This made it more difficult merging my own changed when other members had already changed something which affected my code. In some ways the lack of contribution from other team members on GitHub made it more simple to use so I'm slightly concerned about how much more difficult it will be when we start working in our new group of ten. 
I now know how vital using and having a firm grasp of version control is in the games industry, particularly in large teams. 
\\
When considering a SMART goal for this, I know that committing to a form of version control like GitHub is something I need to be much more comfortable with and as such I intend to make learning it a priority next semester. I intend to go through the further learning exercises on version control before we start our next project as well as get a head start in learning SVN which our group will most likely be using. I will make an SVN account and set up my own project to test out how everything works before the team starts using it.

\section{Self Motivated Learning}

One of the most important skills that I need to develop further involves motivating myself to learn outside of scheduled lessons. Although I did do this to some extent during the game project, this was mostly borne out of necessity. I recognise that in the games industry, self directed learning of new techniques is a must due to the ever evolving nature of technology, as is the drive to do so. 
\\
To begin with I intend on setting myself a SMART goal of completing at least three hours of the C++ tutorials on Pluralsight every week. This should be fairly simple and will help with the Unreal Engine workshops. Using knowledge gained through self directed study to complete small games or projects of my own would also help with motivation. I know that one of the main things that spurred me on to work on our team project was seeing the fruits of our efforts after each iteration of the game. 

\section{Logical Problem Solving}

Potentially the most important ability to possess as a programmer is the ability to use logic and reasoning to solve complex problems. As well as knowing the fundamentals of coding, I need to constantly train myself to think critically. In several of the tasks I have been faced with such as the TIS-100 worksheet, I have had to find tutorials online to help explain some concepts. Although this is part of how we learn, it does feel better and stays in my mind more easily when solutions arrive naturally on my own. 
\\
My SMART goal to remedy this will be to complete SpaceChem and TIS-100 by the end of the semester. Although these are games, they can be incredibly difficult and should help develop my problem solving skills further. I will try to complete at least 2 levels of each game each week.

\section{Conclusion}


The main areas I will be focussing on next semester will include reducing anxiety and increasing the amount of self-directed learning I employ. I will start assignments and essays early and work on them much more consistently throughout the time given. Good planning and time management are key to achieving these goals and my overall plan on becoming a professional games developer hinges on my ability to stay positive during my studies even when things seem overwhelming at first glance. 
\\
Making a conscious effort to ask lecturers and peers more questions or for advice when I need it would also be beneficial, so this will be an ongoing consideration throughout the course. I feel this will help me grow as a developer and will therefore improve the quality of my future projects.

\bibliographystyle{ieeetran}
\bibliography{references}

\end{document}
