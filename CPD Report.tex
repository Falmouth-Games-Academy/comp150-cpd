\documentclass{article}
\usepackage[utf8]{inputenc}
\usepackage{amsmath}

\title{Continuing Personal Development - Report}
\author{jc220866}
\date{December 2018}

\begin{document}
    \pagenumbering{gobble}
    \maketitle
    
\section{Introduction}

No specific end-goal is yet laid out for me, though I do know that I want to make a living through professional game-development as part of a team.
However, throughout the first semester I have noticed numerable occasions where I have lacked a skill essential for programming, both in general and with regards to teamwork.
I have been struggling with motivation, stress is also a possible factor though I lack the ability to notice it.
I'm not proactive in engaging in discussion with my team, nor am I confident in my ability to make a valuable contribution and I always struggle knowing where to begin when faced with a new task.

\newpage

\section{200 Word Challenges}
\subsection{Cognitive}

I do not feel confident in my programming ability.
I believe I lack both the knowledge and experience required to be an efficient component of a game development team.
During the COMP150 game development project, I shied away from contributing when bewildered by the complexity of the code written by my peers.
The correct course of action was to attempt to understand the code and to contribute regardless, however I instead opted to resign and study programming tutorials on my own.

Over the course of the Christmas holiday, I will complete an intermediate level course in C++ within the Unreal Engine.
The website in which the course is hosted has a progress feature with which one can measure their progress through each section and through completing the course as a whole.
Although I already have very basic knowledge of C++, this course assumes no prior knowledge whatsoever and gently increases in difficulty, making it very accessible.
After completing the course, I will have been briefed in all the basic fundamentals of game development within the Unreal Engine, just in time for the multidisciplinary game development project in which we will use said engine.

\subsection{Dispositional}

I've noticed my productivity and motivation dropping severely in the evenings, especially after a day at University.
I often find myself behind on academic tasks that I had planned to do "later".
In retrospect, I believe that the primary issues lie with my time management and my concentration throughout the day.
With no attempt to structure my day or plan my tasks ahead, I frequently under-perform while placing a false positive of productivity on myself, leaving me to feel burned out at the end of the day.

I will create a daily routine which reserves time in the evenings to relax.
My partner and others close to me will review my routine ideas, letting me know their opinion on whether or not it's suitable for myself.
While I've never once attempted to adhere to a routine in my life, I do already find that certain daily rituals are forming even without any effort to structure my day.
With designated leisure time each day, I strongly believe that the designated work time will be filled with more work and less aimlessly bumbling around.
I plan to have the routine drafted in time for the second semester.

\newpage

\subsection{Procedural}

When starting development on a new algorithm, I always struggle with where to begin.
Spending hours staring blankly at empty code will be detrimental to the team, especially when partaking in Agile game-development in a time-constrained professional environment.
When confronted with a problem, I typically assume the task is insurmountable, whereas I should immediately begin planning ways to tackle it on a higher level.
I am aware of many documentation and resources, yet I still feel unable to find answers of any relevance, nor do I notice when I do.

I will create an archive of past and future programs I've written, along with verbose comments explaining each aspect of the code.
My cohorts will have access to the archive. If they find my comments to be of benefit to them then I will know I've achieved my goal.
I have already began the archive. I want to expand upon it with multiple different projects from different sources for years to come.
Having examples and explanations to refer to will help provide a general idea of how to start a new project and how to structure written code.
I aim to collect all code I've written in the past in one sitting, though my archive will be satisfactorily fleshed out upon completion of the C++ course over the Christmas holiday.

\subsection{Interpersonal}

I struggle to work as part of a team when developing a program.
Teamwork is undoubtedly the most important part of professional game development, though it's also the part that bewilders me the most.
I am unsure of the intricacies of coordinating efforts onto a single project, I worry about how we would keep the code structured and free of conflicts.
In previous team projects, I have created new classes for my own algorithms when suitable classes were already present, in fear of causing irreparable damage to the work of others. The correct course of action was to develop an understanding of my peers' code through prompting team discussion.

For this upcoming project, I wish to focus more on developing teamwork skills than actual game development. Therefore, I will endeavour to arrange regular team meetings after each stand-up.
We have the option of measuring attendance if need be.
I already have a decent relationship with the other programmers on my team.
While regular meetings will take away from the total hours available to tackle tasks, I believe it will save us many hours of headache in the long run.
I wish to host the first meeting on the first day of the second semester, immediately after the stand-up.

\newpage

\subsection{Affective}

I've recently been struggling to identify whether or not I am stressed.
Stress left unchecked can manifest into conflict within the workplace and have substantial effect on morale and motivation.\cite{lazarus1995psychological}
While I'm currently unable to place the blame on stress explicitly, I've noticed my overall motivation dropping during busy periods of the day, resulting in less progress being made on game-development projects and tutorials.
Prior to studying at a university level, I had scarcely been exposed to deadlines or self-study. As a consequence of spending prior years in manual labour as opposed to studying academically, I haven't yet fully developed the ability to notice the signs of stress.

I believe I would benefit from assistance in this domain, so I will attend counselling sessions in the second semester, until a satisfactory conclusion is reached. 
In addition to the counselor, my mother and partner will be able to confirm whether any possible conclusions are satisfactory. 
I've never considered myself unstable in this domain, nor do I have any qualms about opening up to others, so there are no foreseeable roadblocks.
Through learning to manage my individual stress, the overall morale of future teams will be higher and there will be less chance of conflict between cohorts and myself.\cite{lazarus1995psychological}
The first session will be held in early January, though I expect significant progress to appear within the following few weeks.

\section{Conclusion}

I have a ways to go before I can function efficiently as part of a professional team, thankfully University offers to teach me more than simply how to program.
Teamwork is a major factor, I will jump-start the development of my teamwork skills by prompting regular interpersonal discussion and making the most out of all future teamwork projects.
New tasks will become easier to tackle when I have an archive of my own code for reference. Through completing the aforementioned C++ course, I am confident I will be able to contribute to my team.
My motivation is becoming easier to read, separating work from leisure with a schedule will further increase my productivity during work hours. Attending counseling sessions too will take weight off of my mind, all while allowing me to develop the skill to notice my stress.

\bibliographystyle{plain}
\bibliography{CPDReferences} 

\end{document}