% Please do not change the document class
\documentclass{scrartcl}

% Please do not change these packages
\usepackage[hidelinks]{hyperref}
\usepackage[none]{hyphenat}
\usepackage{setspace}
%\doublespace so I can read it <- plz remember to remove cuz my reading eyes are incompatible with the assessor

% You may add additional packages here
\usepackage{amsmath}

% Please include a clear, concise, and descriptive title
\title{Using Falmouth Adversity to better myself}

% Please do not change the subtitle
\subtitle{COMP150 - CPD Report}

% Please put your student number in the author field
\author{1707981}

\begin{document}

\maketitle

\section{Introduction}
I meant University.

One day, I'd like to be the lead of an indie game company. First, I need to put my work where my mouth is. Then I must put my mouth out there.

I want to someday create a company around an initial game prototype. Bringing my possible ideas to life is my goal during my time here, and if I can become so confident, to build a team around the better ones.

Here lies personal areas I must improve to do it, and hints of the better person to come out of it. Like a zombie.

\section{Affective - Social Anxiety and Reaching Out}
In the professional industry, it is virtually impossible to be noticed when hiding in a closet. Unless that closet is in an inappropriate location, such as the M27. Contacts are necessary for finding work, so avoiding people is counterproductive to making a mark as an industry leader (and virtually all other positions).

I delay tasks that involve socialising. Ordering my DSA laptop was one example: this required a phone call, which I finally made, perhaps one or two months into the course. My emotions affect my priorities; for example, a DSA laptop is a useful investment but one I neglected. However anxiety holds me back and I would instead `be productive' by working on assignments, neglecting the more useful investment. Since I find it more comfortable overworking than picking up a phone, my anxiety is regularly mistaken for masochism.

This is a fear, so how can I put the A in SMRT? One would be to chain myself to a boulder rolling toward a phone centre. Due to the gravity, the boulder should roll down with it rendering it impossible for me to escape. However, this would also destroy the phone centre, and my bones, so it verges on mildly unachievable.

\section{Interpersonal - Being Inspirational} - Drafted (200 words)

I believe sharing exciting collaborative ideas is symbolic of a good indie team. Unavoidably, the creative lead position requires the ability to coordinate differing ideas.

During Unreal project meetings, few people expressed ideas, possibly for fear of impacting someone else's, or otherwise creating conflict. As one who is often shy to speak out amongst differing ideas, I always wonder whether there are no ideas, or if everyone's suppressing them. We need inspiration--either to make ideas or to show them--and I lack the charisma to inspire them with words. But I believe demonstration has more power than words--it gives teammates a vision of the possibilities.

So before February, I'll compose four short pieces for the Unreal game using LMMS. I'll make a new short theme proposal every Sunday from the 7th Jan until the 28th, ten seconds minimum finishing off any that I like in my own time, and pitching them all in the following meeting.

I believe a great project is one that creatives want to work on even in their free time. Hopefully by doing that myself, regardless of my charisma, they'll be inspired to put their creations forward too. As such, it doesn't matter if mine are good.

\section{Dispositional - Multi-tasking} - Drafted (204 words)

I'm bad at switching tasks. However, I believe extreme focus, rather than intelligent learning, led to my current programming skills, so I preface that I don't want to wipe out the habit entirely. But I'd like to incorporate the `intelligent learning' part.

In Week 7, I spent hours at a time working on my agile essay despite noting in Week 3 the importance of regular breaks. When focused on something, I tend to place full priority on it, especially before anxiety-provoking tasks. This is a bigger issue than overworking overall.

I must intelligently prioritise tasks based on the skills or productions conducive to furthering my creativity, not from my default focus. This is especially true because creative direction oversees a variety of domains, such as writing and art style. This makes it important that I exercise my skills in multiple areas simultaneously.

To practice managing multiple tasks, every Sunday I'll write a schedule of personal projects in the blanks of the course timetable. I'll create two tasks per week, each having a specified minimal time target. One may be coding-related, but the other must not. The minimal time limit for each task may vary but must be an hour or more.

\section{Cognitive - Class Shenanigans and Instantiation} - Drafted (203 words)

In Week 5, I encountered a curiosity that I never fully resolved. In our Object class, I created a `collision' variable, declaring it as a default-initialised CollisionParams object. However, I quickly noticed that object instances were all sharing the same CollisionParams object--and changing each other's parameters! Furthermore, if I initialised it in Object.__init__ instead, inherited classes wouldn't create it.

So I made `collision' optional, and made my code ignore it if it was None. My rationale was that if a programmer forgot to initialise it, their mistake would be obvious. However, the `if' statements piled up--my `solution' missed the mark. Retrospectively, the parent class should be initialised fully by every child class, using super().__init__(). I overlooked this, assuming it cumbersome, to the irony of the if statements' wrath.

Since I resolved this, my new learning goal is slightly different, but relevant. In the past, I avoided virtual functions, because I avoided the new operator. As a gruesomely C-style malloc-happy maniac, I wasn't aware of \textit{new overloading}, but recently learned. In my upcoming C++ coursework, all my dynamically-allocated classes will be instantiated with some form of `new', and I'll use virtual functions, except where there is a notable benefit to the violently messy alternative.

\section{Procedural - It's not Broken until It's Made (Prototyping)} - Drafted (204 words)

During comp150, I worked on the collision system for Frontier. I obsessed over making something precise, for two reasons: one, to exercise a full understanding of the math behind it; and two, I like precision. I'm happy with both; however, in a prototyping scenario, this is not necessary. I'm prone to neglecting this context. I stalled whilst tackling it.

On Michael's advice, I temporarily made a collision system using XY-aligned bounding boxes. I took a break from the understanding the math, instead focussing on finishing a functional prototype. This enabled me to work on the game mechanics. This was successsful, but I'm still prone to losing awareness of context in pursuit of perfection.

So when taking a user story, I'll create a checklist of tasks to achieve the \textit{literal} user story. This addresses \textbf{no details} beyond the user story provided, possibly making some people cry: `You wanted a threatening enemy, you have an enemy. It doesn't do anything. You never asked for it to attack.' Any additional features must not be present for this to work. Instead, they will be added to an `Optional' checklist during production. Precise collision is an example, as the ability to fight can still be achieved without it.

\section{Conclusion}

Write your conclusion here. Though the conclusion should be brief, no more than 100 words, it should do more than merely summarise the report. Focus on the five SMART actions that you intend to take in order to overcome any challenges and/or obstacles. Contextualise how this will help you towards your intended career goal and how this may improve your project for the next semester.

\bibliographystyle{ieeetran}
\bibliography{references}

\end{document}
