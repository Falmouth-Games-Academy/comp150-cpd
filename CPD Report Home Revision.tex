% Please do not change the document class
\documentclass{scrartcl}

% Please do not change these packages
\usepackage[hidelinks]{hyperref}
\usepackage[none]{hyphenat}
\usepackage{setspace}
%\doublespace so I can read it

% You may add additional packages here
\usepackage{amsmath}

% Please include a clear, concise, and descriptive title
\title{Using Falmouth Adversity to better myself}

% Please do not change the subtitle
\subtitle{COMP150 - CPD Report}

% Please put your student number in the author field
\author{1707981}

\begin{document}

\maketitle

\section{Introduction}
I meant University.

One day, I'd like to be the lead of an indie game company. So I need to put my work where my mouth is. Then I must put my mouth out there.

I want to someday create a company around an initial game prototype. Bringing my possible ideas to life is my goal during my time here, and if I can become so confident, to build a team around the better ones.

Here lies personal areas I must improve to do it, and hints of the better person to come out of it. Like a zombie.

\section{Affective - Social Anxiety and Taking Action}
In the professional industry, it is virtually impossible to be noticed when hiding in a closet. Unless that closet is in an inappropriate location, such as the M27. Contacts are necessary for finding work, so avoiding people is counterproductive to making a mark as an industry leader (and virtually all other positions).

Yet I delay tasks that involve initiating social contact. For example, ordering a DSA laptop required a phone call. With long gaps, I didn't complete the process until January due to extreme anxiety. My emotions affect my priorities despite the investment; for example, a DSA laptop would enable me to work anywhere on campus. However, the anxiety made me procrastinate and I would instead be `productive' by working on assignments, neglecting the more useful investment. Since I find it more comfortable overworking than picking up a phone, my anxiety is regularly mistaken for masochism.

I am considering \textbf{joining the drama society}. This is throwing myself into deep end, but no more than the swimming society would. 

Specific    /
Measurable ???
Achievable ???
Relevant    /
Time-bound ???

\section{Interpersonal - Inspiring Others}

To ever be a creative lead, I must convey great ideas in a way that makes the team want to contribute to it. I believe sharing a coherent, exciting, collaborative idea is symbolic of a good indie team.

How do you bring out great ideas in a team of different individuals? During comp150 pre-production, `idea deadlock' occurred: where few people expressed ideas, possibly for fear of treading on someone else's. It could be that some are simply uncreative, or that they were shy to speak out. I sincerely question how often the first could be true, being one who loves making ideas.

Vitally, gameplay was missing until late into the project. Finally, I proposed the DynaSword mechanic. The team approved, perhaps motivated by the lack of anything else. I prototyped it: it was approved, becoming the main. But it feels as though I snuck it in without anyone really asking for it. It wasn't collaboratively developed. However, before I added it, it wasn't developed at all.

Specific   ???
Measurable ???
Achievable ???
Relevant   ???
Time-bound ???

\section{Dispositional - Managing Focus}

A strength and weakness: streakness. I believe extreme focus, rather than intelligent learning, led to my current programming skills.

It causes stress too. In Week 7, I got lost somewhere in my agile essay, spending hours at a time working on it despite noting in Week 3 the importance of regular breaks. When focused on something I tend to place priority on it above all other tasks, especially anxiety-provoking tasks. Contrast to intelligently prioritising tasks based on the skills or productions I need to progress with my career.

I had a similar experience working on the first comp110 worksheet: obsessive focus placing high stakes on an insignificant mark margin. This has a high impact on my life.

I need to improve at switching tasks, so that I have attention toward my own goals rather than the comfort zone of the present task. Hmm...I need a break. Clothes shopping time!

\section{Procedural - Resolving Algorithmic Puzzles}

Usually, when faced with a problem, I prefer to find my own solution. My rationale is that figuring it out front-to-back will leave me a better understanding of it than Googling an answer. However, it takes time, and can block other tasks.

In maybe Week 5-6, Tomas and I worked on basic box collision detection. I was the navigator, and the number of variables coming in: scale, dimensions, testing whether it worked, etc was overwhelming. It would have been best to eliminate the variables and test it with a predefined set. Instead, whilst taking a break upstairs together to reduce the headache, I pulled out a receipt, split it into two boxes and talked  through the testing conditions with a physical demo.

This would have been better had I only done it sooner. When zoned in on the problem, the focus issue occurred. It would have been a good time to stop and take a new approach five minutes from the time we realised we'd stalled, but instead we continued. Deathly determination to solve it didn't bring forth a solution, taking two steps back and starting an impromptu collision detection class did.

Clearly moving . I've noticed that many of my best ideas come from my bathroom breaks. 

\section{Cognitive or Procedural - It's not Broken until It's Made}

I think this is common. In Week [x], I worked on the collision system for Frontier. I obsessed over making something precise, for two reasons: one, to demonstrate a full understanding of the math behind it; and two, I like precision. I'm happy with both; however, I neglected the greater context: a prototype is a demonstration, not a full-functioning product.

However, I stalled. So on Michael's dismayed advice, I temporarily made a collision system using XY-aligned bounding boxes. In effect, I took a break from the understanding the math, instead focussing on finishing a functional game. There I worked on other features I could understand and produce more quickly, and return to my personal collision battle later. Even after making several games (most unfinished), I had finally discovered prototyping.

When taking a user story, I'll create a checklist of three points: First is `plan how', second is `write interfaces/classes', third is `achieve result'. At this stage, I'll address \textbf{no details} beyond the user story provided, possibly making some people cry: `You wanted a threatening enemy, you have an enemy. It doesn't do anything. You never asked for it to attack.'

\section{Conclusion}

Write your conclusion here. Though the conclusion should be brief, no more than 100 words, it should do more than merely summarise the report. Focus on the five SMART actions that you intend to take in order to overcome any challenges and/or obstacles. Contextualise how this will help you towards your intended career goal and how this may improve your project for the next semester.

\bibliographystyle{ieeetran}
\bibliography{references}

\end{document}
