% Please do not change the document class
\documentclass{scrartcl}

% Please do not change these packages
\usepackage[hidelinks]{hyperref}
\usepackage[none]{hyphenat}
\usepackage{setspace}
%\doublespace so I can read it

% You may add additional packages here
\usepackage{amsmath}

% Please include a clear, concise, and descriptive title
\title{A Summary of Personal Experiences in Falmouth Adversity}

% Please do not change the subtitle
\subtitle{COMP150 - CPD Report}

% Please put your student number in the author field
\author{1707981}

\begin{document}

\maketitle

\section{Introduction}
Oops I meant University.

After so long trying to get out of my household, I knew I had a lot of catching up to do with my peers. University has pulled many of my flaws to the surface. To reach my aspiration of being at the creative head of a game project, I have identified several areas to improve: \textbf{Managing anxiety}, \textbf{inspiring people}, \textbf{handling focus}, \textbf{reducing focus}. Through application of my strengths of \textbf{overdedication}, \textbf{inappropriate creativity} and \textbf{obliviousness}, I will attempt to mindlessly brute-force my way to solutions.

Here lies the analysis of a mildly socially-oblivious autistic adult's foray into the social world of programming, writing essays, communicating and drinking, alternately.

Write your introduction here. A brief introduction of about 100 words is recommended, which should state your career goal and the five key skills that you wish to highlight from your weekly reports. When choosing which skills to focus on for this report, be specific. Avoid choosing broad skills that are clearly important for any student, such as \textit{time management} or \textit{communication}. Instead, make it more granular. Consider which specific aspects of these broad areas are a priority for you, personally, and what may have caused or exacerbated the challenge. Tutors are not assessing your knowledge of general study skills. Rather, they are assessing your ability to analyse and reflect on your own learning and personal development as an individual and towards becoming a computing professional.

\section{Affective - Social Anxiety and Taking Action}
In the professional industry, it is virtually impossible to be noticed when hiding in a closet. Unless that closet is in an inappropriate location, such as the M27. Contacts are necessary for finding work, so avoiding people is counterproductive to making a mark as an industry leader (and virtually all other positions).

I delay tasks that involve socialising. Ordering my DSA laptop was one example: this required a phone call, which I finally made, perhaps one or two months into the course. My emotions affect my priorities; for example, a DSA laptop is a useful investment as I could work in the library. However anxiety holds me back and I would instead `be productive' by working on assignments, neglecting the more useful investment. Since I find it more comfortable overworking than picking up a phone, my anxiety is regularly mistaken for masochism.

I am considering \textbf{joining the drama society}. This is throwing myself into deep end, but no more than the swimming society would. 

I will also prank-call Michael every April Fools' day for the next three years, making him the subject of as many memes as possible.

Specific    /
Measurable ???
Achievable ???
Relevant    /
Time-bound ???

\section{Interpersonal - Inspiring Others}

To be a creative lead I must effectively convey an interesting and appealing (hereafter ``awesome'') idea in a way that makes the team want to contribute to it. I believe a coherent, exciting, shared idea is the symbolic of a good indie team.

The question is how to bring out great ideas in a team of unique, different people. During comp150 pre-production, I experienced some `idea deadlock': where few people express unique ideas for fear of treading on someone else's. We had, for example, no gameplay. We just didn't know how to agree on something, let alone something wild and unique.

The great thing however was that we all worked around our differences. Near the end, I proposed the DynaSword mechanic. The team approved, likely motivated on the basis that we had nothing else. I prototyped it and this was approved as well and became the main mechanic. Considering this, I wonder how much demonstrating the tangible thing is more effective than words, especially with unusual or radical ideas?

Specific   ???
Measurable ???
Achievable ???
Relevant   ???
Time-bound ???

\section{Dispositional - Managing Focus}

A strength and weakness: streakness. I believe extreme focus, rather than intelligent learning, led to my current programming skills.

It causes stress too. In Week 7, I got lost somewhere in my agile essay, spending hours at a time working on it despite noting in Week 3 the importance of regular breaks. When focused on something I tend to place priority on it above all other tasks, especially anxiety-provoking tasks. Contrast to intelligently prioritising tasks based on the skills or productions I need to progress with my career.

I had a similar experience working on the first comp110 worksheet: obsessive focus placing high stakes on an insignificant mark margin. This has a high impact on my life.

I need to improve at switching tasks, so that I have attention toward my own goals rather than the comfort zone of the present task. Hmm...I need a break. Clothes shopping time!

\section{Procedural - Resolving Algorithmic Puzzles}

Usually, when faced with a problem, I prefer to find my own solution. My rationale is that figuring it out front-to-back will leave me a better understanding of it than Googling an answer. However, it takes time, and can block other tasks.

In maybe Week 5-6, Tomas and I worked on basic box collision detection. I was the navigator, and the number of variables coming in: scale, dimensions, testing whether it worked, etc was overwhelming. It would have been best to eliminate the variables and test it with a predefined set. Instead, whilst taking a break upstairs together to reduce the headache, I pulled out a receipt, split it into two boxes and talked  through the testing conditions with a physical demo.

This would have been better had I only done it sooner. When zoned in on the problem, the focus issue occurred. It would have been a good time to stop and take a new approach five minutes from the time we realised we'd stalled, but instead we continued. Deathly determination to solve it didn't bring forth a solution, taking two steps back and starting an impromptu collision detection class did.

Clearly moving . I've noticed that many of my best ideas come from my bathroom breaks. 

\section{Cognitive or Procedural - It's not Broken until It's Made}

I think this is common. In Week [x], I worked on the collision system for Frontier. I obsessed over making something precise, for two reasons: one, to demonstrate a full understanding of the math behind it; and two, I like precision. I'm happy with both; however, I neglected the greater context: a prototype is a demonstration, not a full-functioning product.

However, I stalled. So on Michael's dismayed advice, I temporarily made a collision system using XY-aligned bounding boxes. In effect, I took a break from the understanding the math, instead focussing on finishing a functional game. There I worked on other features I could understand and produce more quickly, and return to my personal collision battle later. Even after making several games (most unfinished), I had finally discovered prototyping.

When taking a user story, I'll create a checklist of three points: First is `plan how', second is `write interfaces/classes', third is `achieve result'. At this stage, I'll address \textbf{no details} beyond the user story provided, possibly making some people cry: `You wanted a threatening enemy, you have an enemy. It doesn't do anything. You never asked for it to attack.'

\section{Conclusion}

Write your conclusion here. Though the conclusion should be brief, no more than 100 words, it should do more than merely summarise the report. Focus on the five SMART actions that you intend to take in order to overcome any challenges and/or obstacles. Contextualise how this will help you towards your intended career goal and how this may improve your project for the next semester.

\bibliographystyle{ieeetran}
\bibliography{references}

\end{document}
