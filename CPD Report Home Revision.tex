% Please do not change the document class
\documentclass{scrartcl}

% Please do not change these packages
\usepackage[hidelinks]{hyperref}
\usepackage[none]{hyphenat}
\usepackage{setspace}
%\doublespace so I can read it

% You may add additional packages here
\usepackage{amsmath}

% Please include a clear, concise, and descriptive title
\title{A Summary of Personal Experiences in Falmouth Adversity}

% Please do not change the subtitle
\subtitle{COMP150 - CPD Report}

% Please put your student number in the author field
\author{1707981}

\begin{document}

\maketitle

\section{Introduction}
Oops I meant University.

After so long trying to get out of my household, I knew I had a lot of catching up to do with my peers. University has pulled many of my flaws to the surface. To reach my aspiration of being at the creative head of a game project, I have identified several areas to improve: \textbf{Managing anxiety (and not spreading it!)}, \textbf{managing friendships and acquaintanceships}, \textbf{assertion and conflict}, \textbf{reducing focus}. Through application of my strengths of \textbf{overdedication}, \textbf{inappropriate creativity} and \textbf{obliviousness}, I will attempt to mindlessly brute-force my way to solutions.

Here lies the analysis of a mildly socially-oblivious autistic adult's foray into the social world of programming, writing essays, communicating and drinking, alternately.

Write your introduction here. A brief introduction of about 100 words is recommended, which should state your career goal and the five key skills that you wish to highlight from your weekly reports. When choosing which skills to focus on for this report, be specific. Avoid choosing broad skills that are clearly important for any student, such as \textit{time management} or \textit{communication}. Instead, make it more granular. Consider which specific aspects of these broad areas are a priority for you, personally, and what may have caused or exacerbated the challenge. Tutors are not assessing your knowledge of general study skills. Rather, they are assessing your ability to analyse and reflect on your own learning and personal development as an individual and towards becoming a computing professional.

\section{Affective - Social Anxiety and Taking Action}
In the professional industry, it is virtually impossible to be noticed by hiding in a closet. Unless the closet is in an inappropriate location, such as the eastbound M27. Contacts are key in industry, so any form of social anxiety is counterproductive to making a mark as an industry leader (and all other positions to some degree).

The problem is I have social anxiety. I delay tasks that involve socialising. Ordering my DSA laptop is one example--this required a phone call, which I made, perhaps one or two months into the course. My emotions effect my priorities--for example, a DSA laptop is a useful investment as I could work in the library. However, because of anxiety, I would sooner derive my productivity from working on my assignments than pick up the phone and order it.

I am in progress on a SMART goal for this--\textbf{joining the drama society}. This is throwing myself into deep end, but no more than the swimming society would. 

Specific    /
Measurable ???
Achievable ???
Relevant    /
Time-bound ???

\section{Interpersonal - Inspiring Others}

This is essential. To be a creative lead I must effectively convey an interesting and appealing (hereafter ``awesome'') idea in a way that makes the team want to contribute to it. I believe a coherent, exciting, shared idea is the symbolic of a good indie team.

The problem is how to bring out great ideas in a team of unique, different people. During comp150 pre-production, I experienced some `idea deadlock': where few people express unique ideas for fear of treading on someone else's. We had, for example, no gameplay. We just didn't know how to agree on something, let alone something wild and unique.

The great thing however was that we all worked around our differences. Near the end, I proposed the DynaSword mechanic. The team approved, likely motivated on the basis that we had nothing else. I prototyped it and this was approved as well and became the main mechanic. Considering this, I wonder how much demonstrating the tangible thing is more effective than words, especially with unusual or radical ideas?

Specific   ???
Measurable ???
Achievable ???
Relevant   ???
Time-bound ???

\section{Dispositional - Focus}

A strength and weakness: streakness. Without obsessive focus, I wouldn't have played my own online games 10 years ago. I worry now that I was never as good at it as I was totally obsessed. It was probably somewhere next to can-stacking on my list of autistic traits.

It causes stress too. In Week 7 I got lost somewhere in my agile essay, spending hours at a time working on it despite noting in Week 3 the importance of regular breaks.

Write about 200 words. As above.

\section{Fourth Key Skill}

Write about 200 words. As above.

\section{Fifth Key Skill}

Write about 200 words about. Remember, this is should be reflective and personal to you. Justify the relevance and importance of each of these skills with insight into your personal goals and personal circumstances. Assess your application of the skill throughout the semester and critically reflect on upon their impact it has had on your work and the challenges/obstacles. Acknowledge difficulties. Then, suggest how to overcome the challenge/obstacle in relation to a SMART action. When planning such actions, do not be too general. Consider SMART actions:
specific; measurable; achievable; relevant; and time-bound. Ensure that your proposed action for future development meets all five of these criteria.

\section{Conclusion}

Write your conclusion here. Though the conclusion should be brief, no more than 100 words, it should do more than merely summarise the report. Focus on the five SMART actions that you intend to take in order to overcome any challenges and/or obstacles. Contextualise how this will help you towards your intended career goal and how this may improve your project for the next semester.

\bibliographystyle{ieeetran}
\bibliography{references}

\end{document}
